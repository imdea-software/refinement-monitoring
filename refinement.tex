\section{Observational Refinement}
\label{sec:refinement}


Figure~\ref{fig:treiber} lists a non-blocking stack~\cite{tr/Treiber86} which
stores its elements in a singly-linked list rooted at {\tt Top}, and avoids
blocking lock acquisitions in favor of non-blocking compare-and-swap ({\tt
CAS}) instructions in order to maximize parallelism. In one atomic step, the
{\tt CAS} instruction assigns {\tt Top = n} only if {\tt Top == t}.

\begin{figure}[t]
  \scriptsize
  \begin{minipage}[t]{41mm}
    \begin{verbatim}
struct node *Top;      
      
void push(int v):
  struct node *n,*t;
  n = malloc(sizeof( *n));
  n->data = v;
  do {
    struct node *t = Top;
    n->next = t;
  } while (! CAS (&Top, t, n));
    \end{verbatim}
  \end{minipage}
  \begin{minipage}[t]{40mm}
    \begin{verbatim}
int pop():
  struct node *n,*t;
  do {
    *t = Top;
    if (t==NULL) return EMPTY;
    n = t->next;
  } while (! CAS (&Top, t, n))
  int result = t->data;
  free(t);
  return result;
    \end{verbatim}
  \end{minipage}
  \begin{minipage}[t]{30mm}
    \begin{verbatim}
struct node {
  int data;
  struct node *next;
}
    \end{verbatim}
  \end{minipage}
  \begin{minipage}[t]{25mm}
    \begin{verbatim}
void Thread1():
  push(1);
  int x = pop();
    \end{verbatim}
  \end{minipage}
  \begin{minipage}[t]{25mm}
    \begin{verbatim}
void Thread2():
  int y = pop();
  push(2);
  push(3);
  int z = pop();
    \end{verbatim}
  \end{minipage}
  \caption{An implementation of Treiber's stack. The {\tt pop} operation
  returns the value {\tt EMPTY} when the stack is empty.}
  \label{fig:treiber}
\end{figure}

\begin{figure}[t]
  \footnotesize
  \centering
  %!TEX root = ../draft.tex
\begin{tikzpicture}[node distance=5mm]
  \scriptsize

  \tikzstyle{node}=[minimum size=0pt]
  \tikzstyle{nnode}=[minimum size=0pt,inner sep=0pt]
  \tikzstyle{lnode}=[circle,draw,minimum size=2pt,inner sep=0pt]
  \node[node] (x0)  [] at (0, 0) {};
  \node[lnode] (x1)  [right= .3cm of x0,label={[rotate=60,above=1mm]right:push$(1)$}] {};
  \node[lnode] (x2)  [right of=x1,label={[rotate=60,above=1mm]right:ret}] {}; %label=above:$x$
  \node[lnode] (x3)  [right of=x2,label={[rotate=60,above=1mm]right:pop}] {}; 
  \node[nnode] (x4)  [right of=x3] {}; 
  \node[nnode] (x41) [right= 2.5mm of x4,label={[above=1mm,xshift=15mm]above:{\it preemption before {\tt CAS}}}] {};

  \node[node] (z0)  [right=3mm of x1] {}; 
  \node[node] (z1)  [below=3mm of x1,xshift=5mm] {{\tt n=0xFF}}; 

  \node[node] (y0)  [below=10mm of x3]  {\bf Thread 2};
  \node[lnode] (y1)  [below=12mm of x4,xshift=2mm,label={[rotate=60,above=1mm]right:pop}] {}; 
  \node[lnode] (y2)  [right of=y1,label={[rotate=60,above=1mm]right:ret $\Rightarrow 1$}] {}; 
  \node[lnode] (y3)  [right of=y2,label={[rotate=60,above=1mm]right:${\tt y=1}$}] {}; 
  \node[lnode] (y4)  [right of= y3,label={[rotate=60,above=1mm]right:push$(2)$}] {}; 
  \node[lnode] (y5)  [right of=y4,label={[rotate=60,above=1mm]right:ret}] {}; 
  \node[lnode] (y6)  [right of=y5,label={[rotate=60,above=1mm]right:push$(3)$}] {}; 
  \node[lnode] (y7)  [right of=y6,label={[rotate=60,above=1mm]right:ret}] {}; 

  \node[node] (u0)  [right=1mm of y1] {}; 
  \node[node] (u1)  [below=3mm of y1,xshift=5mm] {{\tt free(0xFF)}}; 

  \node[node] (v0)  [right=2mm of y6] {}; 
  \node[node] (v1)  [below=3mm of y6,xshift=5mm] {{\tt n=0xFF}}; 

  \node[nnode] (x5)  [above=12mm of y7,xshift=2mm] {}; %label=left:$^{8}$
  \node[lnode] (x6)  [right of=x5,label={[rotate=60,above=1mm]right:ret $\Rightarrow 3$}] {}; 
  \node[lnode] (x7)  [right of= x6,label={[rotate=60,above=1mm]right:${\tt x=3}$}] {}; 

  \node[lnode] (y8)  [below=12mm of x7,xshift=3mm,label={[rotate=60,above=1mm]right:pop}] {}; 
  \node[lnode] (y9)  [right of =y8,label={[rotate=60,above=1mm]right:ret $\Rightarrow {\tt EMP}$}] {}; 
  \node[lnode] (y10)  [right of= y9,label={[rotate=60,above=1mm]right:${\tt z=EMP}$}] {}; 

  \node[node] (t1) [right=8mm of x7,yshift=-2mm,label={\bf Thread1}] {};

  \draw[line width=2pt] (x1) -- (x2); %node[draw=none,right] {$h$}
  \draw[] (x2) -- (x3);
  \draw[line width=2pt] (x3) -- (x4);

  \draw[line width=2pt] (y1) -- (y2);
  \draw[] (y2) -- (y3);
  \draw[] (y3) -- (y4);
  \draw[line width=2pt] (y4) -- (y5);
  \draw[] (y5) -- (y6);
  \draw[line width=2pt] (y6) -- (y7);

  \draw[dotted,line width=1pt] (x4) -- (x5); %node[draw=none,right] {$h$}
  \draw[line width=2pt] (x5) -- (x6);
  \draw[] (x6) -- (x7);

  \draw[dotted] (y7) -- (y8);
  \draw[line width=2pt] (y8) -- (y9);
  \draw[] (y9) -- (y10);

  \draw[->] (z0) -- (z1);
  \draw[->] (u0) -- (u1);
  \draw[->] (v0) -- (v1);

\end{tikzpicture}
 \\
  \smallskip
  \parbox{0.8\linewidth}{(a) An execution $e$ of the program; it depicts calls,
  returns, and assignments, and time progresses from left to right.}

  \bigskip
  \begin{minipage}{45mm}
    %!TEX root = ../draft.tex
\begin{tikzpicture}[node distance=4mm]
  \scriptsize

  \tikzstyle{node}=[minimum size=0pt]
  \tikzstyle{nnode}=[minimum size=0pt,inner sep=0pt]
  \tikzstyle{lnode}=[circle,draw,minimum size=4pt,inner sep=0pt,fill]

  \node[lnode] (x1)  [label={left:$\<push>(1)$}] at (0,0) {};

  \node[lnode] (x2)  [right=4mm of x1,yshift=4mm,label={[label distance=1mm]left:$\<pop> \Rightarrow 3$}] {}; 

  \node[lnode] (x3)  [right of=x1,yshift=-4mm,label={[label distance=1mm]left:$\<pop> \Rightarrow 1$}] {}; 

  \node[lnode] (x4)  [right=4mm of x3,label=below:$\<push>(2)$] {};

  \node[lnode] (x5)  [right=4mm of x4,label=right:$\<push>(3)$] {};

  \node[lnode] (x6)  [right of=x5,yshift=6mm,label=above:$\<pop> \Rightarrow {\tt EMPTY}$] {};

  \draw[->,>=stealth',thick] (x1) -- (x2);
  \draw[->,>=stealth',thick] (x1) -- (x3);
  \draw[->,>=stealth',thick] (x3) -- (x4);
  \draw[->,>=stealth',thick] (x4) -- (x5);
  \draw[->,>=stealth',thick] (x5) -- (x6);
  \draw[->,>=stealth',thick] (x2) -- (x6);

\end{tikzpicture}
 \\
    (b) The history of the above execution.
  \end{minipage}
  \hfill
  \begin{minipage}{35mm}
    %!TEX root = ../draft.tex
\begin{tikzpicture}[node distance=4mm]
  \scriptsize

  \tikzstyle{node}=[minimum size=0pt]
  \tikzstyle{nnode}=[minimum size=0pt,inner sep=0pt]
  \tikzstyle{lnode}=[circle,draw,minimum size=4pt,inner sep=0pt,fill]

  \node[lnode] (x1)  [label=left:push$(1)$] at (0,0) {}; %right=4cm of x0,
  \node[lnode] (x2)  [below of=x1,label=left:push$(2)$] {}; 
  \node[lnode] (x3)  [below of=x2,label=left:pop $\Rightarrow 1$] {}; 

  \node[lnode] (x4)  [below of=x3,label=left:push$(3)$] {};

  \node[lnode] (x6)  [right=4mm of x3,label=right:pop $\Rightarrow {\tt EMPTY}$] {};

  \node[lnode] (x5)  [below of=x6,label=right:pop $\Rightarrow \bot$] {};

  \node[lnode] (x51)  [above of=x6,label=right:pop $\Rightarrow\bot$] {};

  \draw[->,>=stealth',thick] (x1) -- (x6);
  \draw[->,>=stealth',thick] (x2) -- (x6);
  \draw[->,>=stealth',thick] (x3) -- (x6);
  \draw[->,>=stealth',thick] (x4) -- (x6);

\end{tikzpicture}
 \\
    (c) A weaker history.
  \end{minipage}

  \caption{An execution and its history.}
  \label{fig:stacks}
\end{figure}

Unfortunately this implementation suffers from a subtle concurrency
bug~\cite{tr/ibm/Michael04} exposed by the two-thread program of
Figure~\ref{fig:treiber} via the execution depicted in
Figure~\ref{fig:stacks}(a). Essentially, Thread~1 wrongfully assumes the
absence of interference from other threads on the successful {\tt CAS}
operation. Thread~1 is preempted right before executing its {\tt CAS} in the
{\tt pop} method; at that moment, its {\tt t} variable points to the first
element in the list at address {\tt 0xFF} added by {\tt push(1)}, and {\tt n ==
NULL}. While Thread~2 updates the list with two additional elements, added by
{\tt push(2)} and {\tt push(3)}, the {\tt t} variable of Thread~1 still points
to the list's first element at address {\tt 0xFF}, which was freed by
Thread~2's call to {\tt pop}, and reallocated in the call to {\tt push(3)}.
When Thread~1 resumes, its {\tt CAS} succeeds, effectively removing two
elements from the list instead of one. The final {\tt pop} of Thread~2 thus
erroneously returns {\tt EMPTY}. Intuitively, this is a problem because the
{\tt EMPTY} value should not have been returned since more elements have been
pushed than popped prior to Thread~2's final {\tt pop} operation. This bug
exposes the fact that our {\tt CAS}-based implementation does not conform to
programmers' expectations of a stack object whose operations execute
atomically. In particular, the assignment {\tt z = EMPTY} should never have
occurred.

Formally this conformance is \emph{observational refinement}. Essentially, a
library implementation $L_1$ \emph{refines} $L_2$ if every observable behavior
of programs using $L_1$ is also observable using $L_2$. This is not the case
between the {\tt CAS}-based implementation $L_1$ of Figure~\ref{fig:treiber}
and an atomic implementation $L_2$, since {\tt z = EMPTY} is observable with
$L_1$ yet not with $L_2$.

We capture the interaction between programs and libraries by their
\emph{histories}, representing the partial happens-before orders of library
method invocations. Fixing arbitrary sets $\mathbb{O}$, $\mathbb{M}$, and
$\mathbb{V}$ of operation identifiers, method names, and parameter/return
values, respectively, we define the set of \emph{operation labels} as
$\mathbb{L} = (\mathbb{M} \times \mathbb{V} \times (\mathbb{V} \cup
\set{\bot}))$. We write $m(u) \Rightarrow v$ to denote the label $\ell =
\tup{m,u,v}$. When $v = \bot$ we say $\ell$ is \emph{pending}, and otherwise
\emph{completed}. A \emph{history} $h = \tup{O,<,f}$ is a partial order $<$ on
a set $O \subseteq \mathbb{O}$ of operations labeled by $f : O \to \mathbb{L}$
such that operations with pending labels are maximal. An operation $o$ with
label $\ell$ is an \emph{$\ell$-operation}, and $o$ is
\emph{pending}/\emph{completed} when $\ell$ is. We say $h$ is \emph{sequential}
when $<$ is a total order on $O$, and \emph{(in)complete} when (not) all
operations are complete. A history set is \emph{sequential} when it contains
only sequential histories.

\begin{example}

  The history of Figure~\ref{fig:stacks}(b) captures the execution of
  Figure~\ref{fig:stacks}(a), where arrows depict the transitive reduction of
  the order relation. Essentially, operations $o_1$ and $o_2$ are ordered if
  $o_1$ happens before $o_2$. For example, although the push$(1)$-operation
  precedes pop $\Rightarrow 3$, since the push$(1)$-operation returns before
  the (pop $\Rightarrow 3$)-operation is called, the (pop $\Rightarrow
  1$)-operation is incomparable to pop $\Rightarrow 3$, since it returns
  after the (pop $\Rightarrow 3$)-operation is called.

\end{example}

This notion of histories gives rise to a natural \emph{weaker-than} relation
$\preceq$ relating any two histories $h_1$ and $h_2$ such that $h_2$ includes
all completed operations of $h_1$, and preserves the order between the
operations common with $h_1$. Pending operations in $h_1$ can be either omitted
or completed in $h_2$. Formally, $\tup{O_1, <_1, f_1} \preceq \tup{O_2, <_2,
f_2}$ if{f} there exists an injection $g: O_2 \to O_1$ such that
\begin{itemize}

  \item $o \in \range(g)$ when $f_1(o) = m(u) \Rightarrow v$ and $v \neq \bot$,

  \item $g(o_1) <_1 g(o_2)$ implies $o_1 <_2 o_2$ for each $o_1, o_2 \in O_2$,

  \item $f_1(g(o)) \ll f_2(o)$ for each $o \in O_2$.

\end{itemize}
where $(m_1(u_1) \Rightarrow v_1) \ll (m_2(u_2) \Rightarrow v_2)$ if{f} $m_1 =
m_2$, $u_1 = u_2$, and $v_1 \in \set{ v_2, \bot }$. When the injection $g$ need
be fixed, we write $h_1 \preceq_g h_2$. We say $h_1$ and $h_2$ are
\emph{equivalent} when $h_1 \preceq h_2$ and $h_2 \preceq h_1$. We do not
distinguish between equivalent histories, and we assume every set $H$ of
histories is closed under inclusion of equivalent histories, i.e.,~if $h_1$ and
$h_2$ are equivalent and $h_1 \in H$, then $h_2 \in H$ as well. Finally,
$\overline{H}$ denotes the closure $\set{ h : \exists h' \in H.\ h \preceq h'
}$ of a history set $H$ under weakening.

\begin{example}

  The history of Figure~\ref{fig:stacks}(c) is weaker than that of
  Figure~\ref{fig:stacks}(b). While one of the two pending pop-operations is
  mapped to the completed (pop $\Rightarrow 3$)-operation, the other is dropped.

\end{example}

We model libraries as sets of histories. Since libraries only dictate methods'
executions between their respective calls and returns, for any execution they
admit, they must also admit executions with weaker inter-operation ordering, in
which calls may happen earlier, and/or returns later. Thus any weakening of a
history admitted by a library must also be admitted. Formally, a \emph{library}
$L$ is a set of histories closed under weakening, i.e.~$\overline{L} = L$, and
a \emph{kernel of $L$} is any set $H$ such that $\overline{H} = L$. A library
$L$ is called \emph{atomic} if it has a sequential kernel. Atomic libraries are
often considered as specifications for concurrent objects. In practice,
libraries can be made atomic by guarding their methods bodies with global lock
acquisitions.

\begin{example}
  \label{ex:atomic_stack}

  The \emph{atomic stack} is the library whose unique kernel is the set of all
  sequential histories for which the return value of each pop operation is
  either the argument value $v$ to the last unmatched push operation, or {\tt
  EMPTY} if there are no unmatched push operations.

\end{example}

We define \emph{observational refinement} between two libraries as history-set
inclusion, saying $L_1$ \emph{refines} $L_2$ if{f} $L_1\subseteq L_2$. Although
this refinement is typically defined with respect to the admissibility of
program executions, recent work shows these definitions are
equivalent~\cite{conf/popl/BouajjaniEEH15}.


