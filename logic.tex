%!TEX root = head.tex

\section{Logical Characterizations of Complete Histories}~\label{sec:logic}

We introduce logical characterizations of complete history sets using existentially-quantified
second-order formulas. Essentially, such a formula expresses the fact that the modeled history
is weaker than another history characterized using a first-order formula. When it describes
the set of complete histories $H_c(L)$ of a library $L$ it states that the modeled history is
weaker than another history in the kernel of $L$.
By the following lemma, this is equivalent
to the fact that the modeled history belongs to $H(L)$.

\begin{lemma}\label{lemma:kernel_histories}

Let $h$ be a history and $L$ a library such that $\ker L$ is defined. Then, 
\[
h\in H(L)\mbox{ iff there exists }h'\in H(\ker L)\mbox{ s.t. }h\preceq h'.
\]

\end{lemma}


%, it uses another symbol $<_{\sf k}$.

\begin{figure}
  \begin{align*}
    o,o' & \in \<Ops>
      \qquad \text{operation-identifier constants} \\
     x & : \<Ops>
      \qquad \text{operation-identifier variables} \\
     \vec{x},\vec{y} & : \<Ops>^*
      \qquad \text{tuples of operation-identifier variables} \\
    @l & \in \<Labels>
      \qquad \text{operation labels} \\[2mm]
    X & ::= o \mid x \\
%    T & ::= i \mid \#(X, i, j) \mid T + T \\
    F & ::= @l(X)\mid X=X\mid \<Matching>(X,X) \mid X \poker X \mid \neg F \mid F \land F \\
    \<PO>(\poker) & ::= \forall x.\ \neg x \poker x  \\
    &\land\,\forall x_1,x_2,x_3.\ (x_1\poker x_2\land x_2 \poker x_3) => x_1 \poker x_3 \\
    < \subseteq \hspace{-1mm}\poker & ::= \forall x_1,x_2.\ x_1 < x_2 => x_1 \poker x_2  \\
    \psi  & ::= \exists \poker.\ \big( < \subseteq \hspace{-1mm}\poker 
    \land \<PO>(\poker)\land \forall \vec{x}\ \exists \vec{y}.\ F\, \big)
  \end{align*}
  \caption{The syntax of $\psi_L$.}
  \label{fig:logic}
\end{figure}

The syntax of the formulas $\psi$ we consider for describing sets of complete histories 
is given in Figure~\ref{fig:logic}. Thus, $\psi$ is a second-order
formula that existentially quantifies over a relation $\poker$ and that is a conjunction of:
\begin{itemize}
  \item a formula $\<PO>(\poker)$ expressing the fact that $\poker$ is a strict
  partial order,
  \item a formula expressing the fact that the order relation $<$ in the modeled history 
  is weaker than the order relation $\poker$,
  \item a formula expressing the fact that the set of operations in the modeled history
ordered by $\poker$ forms a history in the kernel of $L$.
\end{itemize} 
The latter sub-formula is a \emph{closed} formula (i.e., without free variables) 
with $\forall^*\exists^*$ quantifier prefix.

The semantics of $\psi$ is defined as usual. The satisfaction relation $\models$ for quantified formulas
is defined by:
\begin{align*}
h\models \exists \poker.\ \psi' & \mbox{ iff $\exists \poker\subseteq O\times O$ s.t. }\tup{h,\poker}\models \psi' \\
\tup{h,\poker}\models \exists x.\ \varphi & \mbox{ iff $\exists o\in O$ s.t. }\tup{h,\poker}\models \varphi [x \mapsto o]
\end{align*}


\begin{example}

Examples of formulas $\psi_L$ for the most common objects.

\end{example}
