\section{Refinement via Symbolic Reasoning}
\label{sec:logic}

In this section we represent the kernels of typical concurrent objects,
including atomic collections and locks, in a simple first-order language.
Besides function and predicate symbols describing the operation labels and the
order relation of a history, this language includes a predicate {\sf match}
describing a \emph{matching function} $M$ that maps operations $o$ which remove
or test the presence of values to the operations $M(o)$ which added them. For
instance, a matching function $M$ for a history of the atomic stack object is
injective, and maps each (pop $\Rightarrow v_1$)-operation $o$ to a
push$(v_2)$-operation $M(o)$ such that $v_1 = v_2$ when $M(o)$ is defined.
Similarly, the matching function $M$ for an atomic lock object is also
injective, and maps each unlock operation $o$ to a lock operation $M(o)$ if
$M(o)$ is defined.

Figures~\ref{fig:formulas:common}--\ref{fig:formulas:synchronization} list the
properties characterizing typical concurrent objects in a first-order language
whose variables range over operation identifiers, and whose functions and
predicates are interpreted over the operation labels and order relation of a
history, and a given matching function. We use the function symbols {\sf
meth}-od, {\sf arg}-ument, and {\sf ret}-urn, as well as the predicate symbols
{\sf match} and {\sf b}-efore for this purpose. Note that we interpret the
predicate ${\sf match}(x_1,x_2)$ by $o_1 = M(o_2)$ when each $x_i$ binds to
$o_i$. We represent the kernel $H$ of each of the following objects by a
first-order formula $\textsc{Theory}(H)$ such that $h \in H$ if{f} $h, M
\models \textsc{Theory}(H)$, for some $M$, where the satisfaction relation $\_,
\_ \models \_$ is defined as usual, using the aforementioned interpretations.

\begin{figure}

  \footnotesize
  \begin{align*}
    & \textsc{Atomic} \\
    & \forall x.\ \neg {\sf b}(x,x)\land \forall x_1,x_2,x_3.\ ({\sf b}(x_1,x_2) \land {\sf b}(x_2,x_3)) \implies {\sf b}(x_1,x_3) \\
    & \forall x_1,x_2.\ 
      {\sf b}(x_1,x_2) \oplus {\sf b}(x_2,x_1) \land \forall x.\ {\sf ret}(x) \neq \bot 
    \\
    & \textsc{Completed} \\
    & \forall x.\ \neg {\sf b}(x,x)\land \forall x_1,x_2,x_3.\ ({\sf b}(x_1,x_2) \land {\sf b}(x_2,x_3)) \implies {\sf b}(x_1,x_3) \\
    & \forall x.\ {\sf ret}(x) \neq \bot
    \\
    & \textsc{Injective} \\
    & \forall x, y_1, y_2.\ {\sf match}(x, y_1) \land {\sf match}(x, y_2) \implies y_1 = y_2
    \\
    & \textsc{Injective}(X) \\
    & \forall x, y_1, y_2.\ {\sf match}(x, y_1) \land {\sf match}(x, y_2) \land {\sf meth}(y_1)={\sf meth}(y_2)=X \\
    &\quad \implies y_1 = y_2
    \\
    & \textsc{Symmetric} \\
    & \forall x_1,x_2.\ {\sf match}(x_1,x_2) \Leftrightarrow {\sf match}(x_2,x_1)
    \\
    & \textsc{Total}(Y) \\
    & \forall y.\ {\sf meth}(y) = Y \implies \exists x.\ {\sf match}(x,y) \land {\sf b}(x,y)
    \\
    & \textsc{Match1}(X,Y) \\
    & \forall x, y.\ {\sf match}(x,y) \implies 
        {\sf meth}(x) = X \land {\sf meth}(y) = Y \land {\sf arg}(x) = {\sf ret}(y)
    \\
    & \textsc{Match2}(X,Y_1,Y_2) \\
    & \forall x, y.\ {\sf match}(x,y) \implies 
      {\sf meth}(x) = X \land ({\sf meth}(y) = Y_1 \vee {\sf meth}(y) = Y_2) \\
      & \hspace{2.6cm}\land {\sf arg}(x) = {\sf arg}(y)
    \\
    & \textsc{Match3}(X,Y) \\
      & \forall x, y.\ 
        {\sf match}(x,y) \implies {\sf meth}(x) = X \land {\sf meth}(y) = Y
  \end{align*}

  \caption{Generic formulas used across many objects ($\oplus$ denotes exclusive or).}
  \label{fig:formulas:common}

\end{figure}

\begin{example}[Atomic collections]

  The kernel of atomic queue objects is represented by the conjunction of properties stating:
  \begin{itemize}

    \item values are added before they are removed ({\sc AddRem}),

    \item values are removed in the order they are added ({\sc Fifo}), and

    \item remove operations returning empty are not surrounded by matching adds
    and removes ({\sc Empty}).

  \end{itemize}
  We thus represent the kernel $H_{\sf q}$ of atomic queues by
  \begin{align*}
    \textsc{Atomic}  \land \textsc{AddRem}
      \land \textsc{Fifo} \land \textsc{Empty}
  \end{align*}
  Similarly, we represent the kernel $H_{\sf pq}$ of priority queues by
  \begin{align*}
    \textsc{Atomic} \land \textsc{AddRem}
      \land \textsc{Max} \land \textsc{Empty}
  \end{align*}
  and the kernel $H_\mathrm{st}$ of atomic stacks by
  \begin{align*}
    \textsc{Atomic} \land \textsc{AddRem}
      \land \textsc{Lifo} \land \textsc{Empty}
  \end{align*}
  Additionally, we enforce the sanity of the underlying matching function by
  adding the formulas $\textsc{Match1}({\sf add},{\sf
  rem})$ and $\textsc{Injective}$, ensuring removes are matched to adds adding 
  the removed value and that every two removes are matched to different adds.

\end{example}

\begin{figure}

  \footnotesize
  \begin{align*}
    & \textsc{AddRem} \\
    & \forall r.\ {\sf meth}(r) = {\sf rem} \land {\sf ret}(r) \neq {\sf empty} \implies
      \exists a.\ {\sf match}(a,r) \land {\sf b}(a,r)
    \\
    & \textsc{Empty} \\
    & \forall e, a.\ {\sf meth}(e) = {\sf rem} \land {\sf ret}(e) = {\sf empty} \land {\sf meth}(a) = {\sf add} \\
    & \quad \land {\sf b}(a,e) \implies \exists r.\ {\sf match}(a,r) \land {\sf b}(r,e)
    \\
    & \textsc{Fifo} \\
    & \forall a_1, a_2, r_2.\ {\sf meth}(a_1) = {\sf add} \land {\sf match}(a_2,r_2) \\
    & \quad \land {\sf b}(a_1,a_2) \implies \exists r_1.\ {\sf match}(a_1,r_1) \land {\sf b}(r_1,r_2)
    \\
    & \textsc{Lifo} \\
    & \forall a_1, a_2, r_1.\ {\sf meth}(a_2) = {\sf add} \land {\sf match}(a_1,r_1) \\
    & \quad \land {\sf b}(a_1,a_2) \land {\sf b}(a_2,r_1) \implies \exists r_2.\ {\sf match}(a_2,r_2) \land {\sf b}(r_2,r_1)
    \\
    & \textsc{Max} \\
    & \forall a_1, a_2, r_1.\ {\sf meth}(a_2) = {\sf add} \land {\sf match}(a_1,r_1) \land {\sf b}(a_2,r_1) \\
    & \quad \land {\sf arg}(a_1) < {\sf arg}(a_2) \implies \exists r_2.\ {\sf match}(a_2,r_2) \land {\sf b}(r_2,r_1)
  \end{align*}

  \caption{Formulas for collection objects.}
  \label{fig:formulas:collecions}

\end{figure}

\begin{example}[Atomic sets]

  Unlike the atomic queues and stacks which behave as \emph{multisets} and
  return removed values, the atomic set’s remove method takes as an
  argument a value to be removed, and succeeds whether or not the value is
  present. The formulas {\sc Include} and {\sc Exclude} specify when a 
  contains-operation may return true. We represent the kernel $H_\mathrm{s}$
  of atomic sets by  
  \begin{align*}
    \textsc{Atomic} \land \textsc{Include} \land \textsc{Exclude}
  \end{align*}
  Additionally, we enforce the sanity of the underlying matching function by
  adding the formulas: $\textsc{Match2}({\sf add},{\sf
  remove},{\sf contains})$ ensuring removes and contains returning true
  are matched to adds of the same value, $\textsc{Injective}({\sf remove})$ 
  ensuring that every two removes are matched to different adds,
  $\textsc{AddRem}$ ensuring that the matching function maps removes to preceding
  adds, and 
  $\textsc{MatchingAdd}$ (resp., $\textsc{MatchingRem}$) ensuring that every add matched to a remove
  (resp., remove matched to an add) is the first in a sequence of adds adding 
  (resp., removes removing) the same value.

\end{example}

\begin{figure}

  \footnotesize
  \begin{align*}
    & \textsc{Include} \\
    & \forall c.\ {\sf meth}(c) = {\sf contains} \land {\sf ret}(c) = {\sf true} \\
    & \quad \implies \exists a.\ {\sf match}(c,a) \land {\sf b}(a,c) \\
    & \forall c,a,r.\ {\sf meth}(c)={\sf contains} \land {\sf ret}(c) = {\sf true} \\
    & \quad \land {\sf meth}(r) = {\sf remove}\land {\sf meth}(a) = {\sf add} \land  {\sf match}(c,a) \land {\sf match}(r,a)  \\
    & \quad \implies {\sf b}(a,c) \land {\sf b}(c,r)
    \\
    & \textsc{Exclude} \\
    & \forall c,a.\ {\sf meth}(c)={\sf contains} \land {\sf ret}(c) = {\sf false} \\
    & \quad \land {\sf meth}(a) = {\sf add}\land {\sf arg}(c) = {\sf arg}(a) \land {\sf b}(a,c) \\
    & \quad \implies \exists r.\ {\sf meth}(r)={\sf remove} \land {\sf match}(r,a) \land {\sf b}(r,c)
    \\
    & \textsc{AddRem} \\
    & \forall a,r.\ {\sf match}(a,r) \land {\sf meth}(r) = {\sf remove}  \implies {\sf meth}(a)={\sf add} \land {\sf b}(a,r) 
    \\
    & \textsc{MatchingAdd} \\
    & \forall a,x,p.\ {\sf match}(a,x) \land {\sf arg}(p) = {\sf arg}(a) \\
    & \quad \land \forall q.\ {\sf b}(q,p) \vee {\sf b}(a,q) \vee {\sf arg}(q)\neq {\sf arg}(a) \\
    & \quad \implies {\sf meth}(p)={\sf remove} \vee ({\sf meth}(p)={\sf contains} \land {\sf ret}(p) = {\sf false}) 
    \\
    & \textsc{MatchingRem} \\
    & \forall a,r,p.\ {\sf match}(r,a) \land {\sf arg}(p) = {\sf arg}(r) \land {\sf meth}(r)={\sf remove}\\
    & \quad \land \forall q.\ {\sf b}(q,p) \vee {\sf b}(r,q) \vee {\sf arg}(q)\neq {\sf arg}(r) \\
    & \quad \implies {\sf meth}(p)={\sf add} \vee ({\sf meth}(p)={\sf contains} \land {\sf ret}(p) = {\sf true}) 
  \end{align*}

  \caption{Formulas for set objects.}
  \label{fig:formulas:set}

\end{figure}

\begin{example}[Atomic register]

  Atomic registers with read and write methods essentially ensure
  that each value read is written by the most recent {\sc write}-operation.
  We represent the kernel $H_\mathrm{r}$ of atomic registers by
  \begin{align*}
    \textsc{Atomic} \land \textsc{ReadWrite} \land \textsc{ReadFrom}
  \end{align*}
  and enforce the sanity of the underlying matching function by adding the
  formulas $\textsc{Injective}$ and $\textsc{Match1}({\sf write},{\sf read})$,
  ensuring reads are matched to writes writing the read value.

\end{example}

\begin{figure}

  \footnotesize
  \begin{align*}
    & \textsc{ReadWrite} \\
    & \forall r.\ {\sf meth}(r) = {\sf read} \implies \exists w.\ {\sf match}(w,r) \land {\sf b}(w,r)
    \\
    & \textsc{ReadFrom} \\
    & \forall w_1, r.\ {\sf meth}(w_1) = {\sf write} \land {\sf meth}(r) = {\sf read} \land \neg {\sf match}(w_1,r) \\
    & \quad \land {\sf b}(w_1,r) \implies \exists w_2.\ {\sf match}(w_2,r) \land {\sf b}(w_1,w_2) \land {\sf b}(w_2,r)
  \end{align*}

  \caption{Formulas for register objects.}
  \label{fig:formulas:register}

\end{figure}

\begin{example}[Synchronization objects]

  Atomic lock objects with lock and unlock methods ensure that at
  most one thread holds a lock at any moment. We represent the kernel of atomic
  locks by
  \begin{align*}
    \textsc{Atomic} \land \textsc{Total}({\sf unlock}) \land \textsc{Mutex}
  \end{align*}
  and enforce coherent matching by adding $\textsc{Match3}({\sf lock, unlock})$
  and $\textsc{Injective}$ formulas. Atomic semaphore objects with acquire and
  release methods generalize atomic locks, ensuring that at most $n$ copies of
  a resource are held at any moment, for some fixed $n \in \mathbb{N}$. We
  represent the kernel of atomic semaphores by
  \begin{align*}
    \textsc{Atomic} \land \textsc{Total}({\sf release}) \land \textsc{Limit}
  \end{align*}
  and enforce coherent matching by adding $\textsc{Match3}({\sf acquire,
  release})$ and $\textsc{Injective}$ formulas.
  
  Exchanger objects are used to pair up threads so they can atomically swap
  values. The only method of this object is exchange, which receives as
  input a value $v$ that a thread it offers to swap and returns a value $v'\neq
  {\tt null}$ if it has paired up with an exchange$(v')$ operation. The latter
  operation will return the value $v$. We represent the kernel of exchanger
  objects by
  \begin{align*}
     \textsc{Completed} \land \textsc{Exchange}
  \end{align*}
  The kernel of this object contains non-sequential histories
  because the time spans of exchange
  operations that pair up overlap.
  Additionally, we enforce the sanity of the underlying matching function by
  adding the formulas: $\textsc{Match1}({\sf exchange},{\sf
  exchange})$, $\textsc{Injective}$, and $\textsc{Symmetric}$
  ensuring that the matching function is injective and symmetric and that it
  associates operations returning a value $v$ to operations that receive $v$
  as input, and $\textsc{MatchOverlap}$ ensuring that matched operations
  overlap in time.
  
\end{example}

\begin{figure}

  \footnotesize
  \begin{align*}
    & \textsc{Mutex} \\
    & \forall \ell_1, \ell_2.\ {\sf meth}(\ell_1) = {\sf meth}(\ell_2) = {\sf lock} \land {\sf b}(\ell_1,\ell_2) \\
    & \quad \implies \exists u.\ {\sf meth}(u) = {\sf unlock} \land {\sf b}(\ell_1,u) \land {\sf b}(u,\ell_2)
    \\
    & \textsc{Limit} \\
    & \forall x_0,\dots,x_n.\ \bigwedge_{0 \leq i < n} {\sf b}(x_i,x_n) \land \bigwedge_{0 \leq i \leq n} {\sf meth}(x_i) = {\sf acquire} \\
    & \quad \implies \exists r.\ {\sf b}(r,x_n) \land \bigvee_{0 \leq i \leq n} {\sf match}(y,x_i)
    \\
    & \textsc{Exchange} \\
    & \forall x.\ {\sf ret}(x) \neq {\sf null} \implies  \exists y.\ {\sf match}(x,y)  \\
    & \textsc{MatchOverlap} \\
    & \forall x_1,x_2.\ {\sf match}(x_1,x_2) \implies \neg {\sf b}(x_1,x_2) \land \neg {\sf b}(x_2,x_1) 
  \end{align*}

  \caption{Formulas for synchronization objects.}
  \label{fig:formulas:synchronization}

\end{figure}
