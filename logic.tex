%!TEX root = head.tex

\section{Logical Characterizations of Complete Histories}

In this work, we consider a logical formulation of checking whether a given \emph{complete} history $h$ belongs to a set of histories $H(L)$,
which is based on a formula $\psi_L$ describing $H(L)$. This formula belongs to the theory of labeled partially ordered sets,
that uses a symbol $<$ for the partial order relation defined by a history and the symbols in $\<Labels>$ for the
labels of the history's operations.

The formula $\psi_L$ describing the set of complete histories $H_c(L)$ of a library $L$ states that 
a history $h$ belongs to $H_c(L)$ iff it is weaker than the history of an execution in the 
\emph{kernel} of $L$. This is possible because of the following result.

\begin{lemma}

Let $h$ be a history and $L$ a library with kernel $\ker L$. Then, 
\[
h\in H(L)\mbox{ iff there exists }h'\in H(\ker L)\mbox{ s.t. }h\preceq h'.
\]

\end{lemma}


%, it uses another symbol $<_{\sf k}$.
More precisely, $\psi_L$ is a formula of the form:
\begin{align}\label{eq:formula}
\psi_L ::= F\land \forall o,o'.\ o < o' => o<_k o',
\end{align}
where $F$ is a formula describing histories of executions in the kernel of $L$. 
The order relation in a history from $H(\ker L)$ is denoted by $<_k$.
Formally, $F$ is a \emph{closed} formula in the \emph{History Logic} described 
in Figure~\ref{fig:logic}.


\begin{figure}
  \begin{align*}
    o,o' & \in \<Ops>
      \qquad \text{operation-identifier constants} \\
     x & : \<Ops>
      \qquad \text{operation-identifier variables} \\
    @l & \in \<Labels>
      \qquad \text{operation labels} \\[2mm]
    X & ::= o \mid x \\
%    T & ::= i \mid \#(X, i, j) \mid T + T \\
    A & ::= @l(o)\mid \<Matching>(X,X) \mid X <_k X \mid \neg A \mid %\\
    %&\hspace{6mm}  
    A \land A \\
    F  & ::= \forall \vec{x}\ \exists \vec{y}.\ A
  \end{align*}
  \caption{The syntax of History Logic.}
  \label{fig:logic}
\end{figure}

\begin{example}

Examples of formulas $\psi_L$ for the most common objects.

\end{example}
