\section{Discussion}
\label{sec:discussion}

In this work, we do not claim \emph{theoretical completeness} of the {\sc
saturate} algorithm since we lack a formal completeness proof. Such a proof
appears to be very challenging, and would seem to rely on yet-to-be-articulated
assumptions on naturally-occurring concurrent objects. However, our empirical
experience, reported in Section~\ref{sec:exp}, suggests completeness, and we
have no evidence to suggest that {\sc saturate} is incomplete, even with
operation removal.

Based on our experience in this and prior works~\cite{ conf/esop/BouajjaniEEH13,
conf/popl/BouajjaniEEH15}, we conjecture the theoretical completeness of {\sc
saturate} as well, with and without operation removal, for naturally-occurring
objects like atomic stacks, queues, and locks, under the assumption that all
operations eventually return. This would imply that the propositional
backtracking thought to be inherent in linearizability is unnecessary, and that
linearizability is polynomial-time checkable for typical concurrent objects.
This insight has not been suggested by any previous works of which we are aware.

While the {\sc enumerate} algorithm is complete by definition, it follows from
Sections~\ref{sec:logic} and~\ref{sec:propositional} that the {\sc symbolic}
algorithm is also complete without operation removal, and it follows form
Section~\ref{sec:obsolete} that both {\sc enumerate} and {\sc symbolic} are
incomplete with operation removal.
