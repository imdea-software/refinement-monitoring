%!TEX root = head.tex

\section{Monitoring History Inclusion}

\subsection{Monitoring complete histories}

Though $\psi$ is a second-order formula, the satisfaction problem of whether a complete history $h$ satisfies
$\psi$ can be reduced to a propositional satisfiability problem.
Thus, given a complete history $h=\tup{O,<,f}$, let $\varphi(h)$ be a formula describing the order constraints
and the labels in $h$:
\[
\varphi(h)::= \bigwedge_{o\in O} f(o)\land \bigwedge_{o,o'\in O, o<o'} o< o'\land \<PO>(<)
\]

Also, let $\<Dom>(O)::=\forall x. x\in O$ be a formula restricting the interpretation domain of every 
operation-identifier variable to the elements of $O$. The formula $x\in O$ is a syntactic sugar for 
$
\bigvee_{o\in O} x=o
$.

%Also, given $h$, let $F[h]$ be the HL formula $F$ where the relation symbol $<_k$ is replaced by $<$
%and the quantifiers are instantiated on the set of operations $O$, 
%i.e., every sub-formula $\exists x.\ G$ of $F$ is rewritten into
%\[
%\bigvee_{o\in O} G[x\mapsto o].
%\]

\begin{theorem}\label{th:satisfiability}
Let $h$ be a complete history and $\psi::=\exists\poker.\ \psi'$ a formula as in Figure~\ref{fig:logic}. Then,
\[
h\models \psi\mbox{ iff }\varphi(h)\land \<Dom>(O)\land \psi'\mbox{ is satisfiable.}
\]
\end{theorem}
\begin{proof}
($=>$) Let $h=\tup{O,<,f}$. Since $h\models \psi$, there exists a relation $\poker\subseteq O\times O$
such that $\tup{h,\poker}\models\psi'$, which implies that $\tup{h,\poker}$ is a model of $\varphi(h)\land \psi'$.

($\Leftarrow$) Let $\tup{h',\poker}$ be a model of $\varphi(h)\land \<Dom>(O)\land \psi'$. By the definition of $\varphi(h)$, 
we obtain that $h\preceq h'$. Since $\psi'$ requires that $\poker$ is stronger than the order relation in $h'$,
we obtain that $\poker$ is also stronger than the order relation in $h$. Moreover, since $\tup{h',\poker}$ is a model
of $ \<Dom>(O)$, we get that $\poker$ is an order relation between elements of $O$.
Therefore, $\tup{h,\poker}$ is also a model of $\psi'$, which implies that $h$ is a model of $\psi$.
\end{proof}

The satisfiability of $\varphi(h)\land \<Dom>(O)\land \psi'$ can be reduced to propositional satisfiability since
the domain of the operation-identifier variables is fixed and the quantifiers can be replaced by finite 
conjunctions/disjunctions, e.g., any sub-formula $\exists x.\ \varphi$ can be replaced by
\[
\bigvee_{o\in O}\ \varphi[x\mapsto o].
\]

\begin{corollary}\label{cor:satisfiability}

State the reduction to propositional satisfiability

\end{corollary}

\subsection{Monitoring histories with pending operations}

The formulas $\psi$ defined in Section~\ref{sec:logic} describe only complete histories
of a library $L$ but, they can be used to check even whether a history $h$ with \emph{pending} operations 
belongs to $H(L)$. Essentially, one can enumerate all the possible ways of completing or removing
pending operations of $h$ and check whether the obtained histories satisfy the formula $\psi$. 
In general, this works if the kernel of $L$ is defined and it contains only executions with completed operations.

\begin{definition}

Let $h_1=\tup{O_1,<_1,f_1}$ and $h_2=\tup{O_2,<_2,f_2}$ be two histories. We say $h_2$ 
is a \emph{completion} of $h_1$, written $h_1\preceq_c h_2$, when

\begin{itemize}

  \item $h_2$ consists only of completed operations, i.e., for all $o\in O_2$, $f_2(o)=m(u)=>v$ and $v\neq \bot$, and 
  
  \item there exists an injection $g:O_2 -> O_1$ satisfying the properties in Definition~\ref{def:weaker_than}
  and $o_1<_2 o_2$ implies $g(o_1)<_2 g(o_2)$ for each $o_1,o_2\in O_2$.

\end{itemize}

\end{definition}

\begin{example}

Examples of completions.

\end{example}

\begin{lemma}\label{lemma:pending_histories}

Let $L$ be a library such that $\ker L$ is defined and the operations of any execution in $\ker L$ are completed.
A history $h$ belongs to the histories $H(L)$ of a library $L$ iff there exists 
a history $h_c$ such that $h\preceq_c h_c$ and $h_c\in H(L)$.

\end{lemma}

\begin{proof}

($=>$) Let $h=\tup{O,<,f}$. By Lemma~\ref{lemma:kernel_histories}, $h\in H(L)$ implies that there exists 
a complete history $h'=\tup{O',<',f'}$ such that $h'\in H(\ker L)$ and $h\preceq h'$. 
Then, there exists an injection $g:O' -> O$ 
satisfying the properties in Definition~\ref{def:weaker_than}. We define $h_c=\tup{O',<_c,f'}$ 
such that $o<_c o'$ iff $g(o) < g(o')$. Clearly, $h\preceq_c h_c$ and $h_c\preceq h'$. The latter implies
$h_c\in H(L)$, which finishes the proof.

($\Leftarrow$) Since $h\preceq_c h_c$ implies $h\preceq h_c$, then by Lemma~\ref{downward-closed},
$h_c\in H(L)$ implies $h\in H(L)$.
\end{proof}

Though we seek to develop an efficient monitor that can check whether a history with pending
operations is a refinement violation, the result in Lemma~\ref{lemma:pending_histories}
leads to an exponential blow-up: the number of possible completions of a given history $h$
is in general exponential in the number of operations in $h$. Therefore, we introduce
a sound, but incomplete, decision procedure that avoids enumerating all completions of $h$.
Essentially, this works by constructing a formula which is weaker than all the first-order formulas
corresponding to completions of $h$ via Theorem~\ref{th:satisfiability}. Therefore, if this formula
is unsatisfiable, then there exists no completion of $h$ which is included in $H(L)$.

Thus, given a history $h=\tup{O,<,f}$, let $O_c\subseteq O$ be the set of completed operations
in $h$, i.e., 
\[
O_c=\set{o:f(o)=m(u)=>v,v\neq\bot}.
\]

An operation $o\in O$ is called \emph{obsolete} iff there exists no pending operation $o'\in O$ which 
overlaps with $o$, i.e., for all $o'\in O$, $\neg o<o'$ and $\neg o'<o$ implies
$f(o')\in O_c$. The set of obsolete operations in the history $h$ is denoted by $O_{ob}$.

\begin{lemma}\label{lemma:pref_obsolete}

Let $L$ be a library such that $\ker L$ is defined and the operations of any execution in $\ker L$ are completed.
Let $h_1=\tup{O_1,<_1,f_1}\in H(L)$ and $Ob$ a set of obsolete operations in $h_1$. Then, there exists a 
history $h_2=\tup{O_2,<_2,f_2}$ such that 

\begin{itemize}

	\item $h_2$ is a prefix of $h_1$ and $h_2\in H(L)$, 

	\item $h_2$ includes all operations in $Ob$ and no operation ordered by $h_1$ after all operations in $Ob$, i.e.,
	$Ob\subseteq O_2$ and for all $o\in O_2$ there exists $o'\in Ob$ such that $\neg o'<o$.
	
\end{itemize}

\end{lemma}

\begin{proof}

Since $h_1\in H(L)$, there exists a completion $h_{1,c}$ and a kernel history $h_3\in H(\ker L)$ such that
$h_1\preceq_c h_{1,c} \preceq h_3=\tup{O_3,<_3,f_3}$. Let $h_4=\tup{O_4,<_4,f_4}$ be a minimal prefix of $h_3$ that
includes all operations in $Ob$. We define $O_2=O_4$, $<_2=<_1 |_{O_4}$, and $f_2=f_1 |_{O_4}$.

Since $h_4$ is downward closed w.r.t. $<_3$ and any two operations in $O_4$ ordered by $<_1$ are 
also ordered by $<_3$, it follows that $h_2$ is downward closed w.r.t. $<_1$. 
%Every $o\in O_4$ is completed in $h_1$: 
Therefore, $h_2$ is a prefix of $h_1$.

Every operation $o\in O_4$ is completed in $h_1$: by definition, 
$h_1$ orders every obsolete operation before every pending operation $o'$,
and by the minimality of $h_4$, it follows that $O_4$ doesn't include pending operations
of $h_1$. This implies that $f_1 |_{O_4}=f_4$ which is enough to conclude that 
$h_2\preceq h_4$. By Lemma~\ref{lemma:kernel_histories_prefix}, $h_4\in H(\ker L)$ and consequently, 
$h_2\in H(L)$.

Assume by contradiction that $O_4$ includes an operation $o$ such that $o' <_1 o$ for each $o'\in Ob$.
Since $h_1\preceq h_3$, we obtain that $o' <_3 o$ for each $o'\in Ob$, which contradicts the minimality of
the prefix $h_4$.
\end{proof}


Let $\varphi_c(h)$ be a formula describing the order constraints
and the labels in $h$ but only for completed operations:
\[
\varphi_c(h)::= \bigwedge_{o\in O_c} f(o)\land \bigwedge_{o,o'\in O_c, o<o'} o< o'\land \<PO>(<)
\]
%The formula $\<PO>_c$ restricts the interpretation of the universally-quantified variables to
%complete operations:
%\<PO>_c(<)

Moreover, given a formula $\varphi::=\forall \vec{x}\ \exists \vec{y}.\ F$, where 
$F$ is defined as in Figure~\ref{fig:logic}, we define the formula $\varphi_{ob}$ that
restricts the interpretation of the universally quantified variables to obsolete operations
and the interpretation of the existentially quantified variables to complete operations:
\begin{align}
\varphi_{ob}::=\forall \vec{x}\ \exists \vec{y}.\ \bigwedge_{x\in\vec{x}}\ x\in O_{ob} => \big(\bigwedge_{y\in\vec{y}}\ y\in O_c \land F\big).
\end{align}

%\exists \poker.\ \big( < \subseteq \hspace{-1mm}\poker 
%    \land \<PO>(\poker)\land \forall \vec{x}\ \exists \vec{y}.\ F
    
Given a formula $\psi::=\exists\poker.\ \psi'$ as in Figure~\ref{fig:logic}, let $\psi'_{ob}$ be the formula
obtained from $\psi'$ by replacing the sub-formula $\varphi::=\forall \vec{x}\ \exists \vec{y}.\ F$
with $\varphi_{ob}$.

\begin{theorem}\label{th:satisfiability_pending}

Let $L$ be a library such that $\ker L$ is defined and the operations of any execution in $\ker L$ are completed.
Also, let $\psi::=\exists\poker.\ \psi'$ be a formula as in Figure~\ref{fig:logic} describing the set of 
complete histories $H_c(L)$. Then,
\[
h\in H(L)\mbox{ implies }\varphi_c(h)\land \psi'_{ob}\mbox{ is satisfiable.}
\]

\end{theorem}

\begin{proof}

Let $h=\tup{O,<,f}$. By Lemma~\ref{lemma:pending_histories}, there exists $h_1=\tup{O_1,<_1,f_1}$  %and~\ref{lemma:kernel_histories}
a completion of $h$  such that %and $h'\in H(\ker L)$
$h\preceq_c h_1$ and $h_1\in H(L)$. Also, by Theorem~\ref{th:satisfiability}, we get that
\[
\varphi(h_1)\land \<Dom>(O_1)\land \psi'\mbox{ is satisfiable}.
\]

We show that the following entailment holds
\begin{align}\label{eq:entailment}
\big(\varphi(h_1)\land \<Dom>(O_1)\land \psi'\big) => (\varphi_c(h)\land \psi'_{ob}),
\end{align}
which implies that the right-hand side of the entailment is also satisfiable.

Let $\tup{h_2,\poker}$ be a model of the left-hand side formula. Since all
the constraints in $\varphi_c(h)$ are included in $\varphi(h_1)$ (as a completion of $h$, 
$h_1$ preserves the complete operations in $h$ with the same labeling and the same
order constraints), $\tup{h_2,\poker}$ is a model of $\varphi_c(h)$. 

Let $\varphi::=\forall \vec{x}\ \exists \vec{y}.\ F$ and
$
\psi'::=< \subseteq \hspace{-1mm}\poker 
\land \<PO>(\poker)\land \varphi
$. We prove that 
\[
\tup{h_2,\poker}\models \psi'_{ob}::=< \subseteq \hspace{-1mm}\poker 
\land \<PO>(\poker)\land \varphi_{ob}.
\]
The first two conjuncts of $\psi'_{ob}$ are clearly satisfied by $\tup{h_2,\poker}$ since
they are also included in $\psi'$.

Since $h_2\in H(L)$, by Lemma~\ref{lemma:pref_obsolete}, there exists a prefix $h_3=(O_3,<_3,f_3)$
of $h_2$ that includes all the obsolete operations of $h$ and no pending operation of $h$
such that $h_3\in H(L)$. Therefore, there exists $\poker'$ such that $(O_3,\poker',f_3)$ is
a prefix of $(O_2,\poker,f_2)$ and 
\[
\tup{h_3,\poker'}\models \varphi(h_3)\land \<Dom>(O_3)\land \psi',
\]
and in particular, $\tup{h_3,\poker'}\models \<Dom>(O_3)\land \varphi$. The latter
implies that
\[
\tup{h_3,\poker'}\models 
\forall \vec{x}\ \exists \vec{y}.\ \bigwedge_{x\in\vec{x}}\ x\in O_3 => \big(\bigwedge_{y\in\vec{y}}\ y\in O_3 \land F\big).
\]

Let $O_{ob}$ be the set of obsolete operations of $h$. % and $O_c$ the set of complete operations of $h$. 
Since $O_{ob}\subseteq O_3$, we get that 
\[
\tup{h_3,\poker'}\models 
\forall \vec{x}\ \exists \vec{y}.\ \bigwedge_{x\in\vec{x}}\ x\in O_{ob} => \big(\bigwedge_{y\in\vec{y}}\ y\in O_3 \land F\big).
\]
Now, because $h_3$ is a prefix of $h_2$ and $(O_3,\poker',f_3)$ is a prefix of $(O_2,\poker,f_2)$, we get that
\[
\tup{h_2,\poker}\models 
\forall \vec{x}\ \exists \vec{y}.\ \bigwedge_{x\in\vec{x}}\ x\in O_{ob} => \big(\bigwedge_{y\in\vec{y}}\ y\in O_3 \land F\big).
\]

Let $O_c$ be the set of complete operations of $h$. Since $O_3\subseteq O_c$, we get that
\[
\tup{h_2,\poker}\models 
\forall \vec{x}\ \exists \vec{y}.\ \bigwedge_{x\in\vec{x}}\ x\in O_{ob} => \big(\bigwedge_{y\in\vec{y}}\ y\in O_c \land F\big).
\]
Therefore, $\tup{h_2,\poker}\models \varphi_{ob}$, which concludes the proof.
\end{proof}


\begin{corollary}\label{cor:satisfiability}

State the reduction to propositional satisfiability

\end{corollary}

\begin{example}

Show that the reduction in Theorem 2 is incomplete. A history with pending operations which is not linearizable but
the formula is satisfiable. 

One hopes that the history is extensible, that all pending operations become completed and then, detect the violation.

\end{example}

\subsection{Removing Completed Matching Clusters}

\begin{theorem}

Let $L$ be a library for which $\ker L$ is defined. If $H(\ker L)$ is $\<Matching>$-closed, 
then $H(L)$ is also $\<Matching>$-closed.

\end{theorem}

\begin{proof}

Let $h$ be a history of $H(L)$. 
By Lemmas~\ref{lemma:kernel_histories} and~\ref{lemma:pending_histories}, 
there exists a completion $h_{c}$ and a kernel history $h'\in H(\ker L)$ such that
$h\preceq_c h_{c} \preceq h'$. Also, let $O_1$ be a matching cluster of $h$.
Since all the operations in $O_1$ are completed, they are included in $h'$ with the
same labeling. By hypothesis, $H(\ker L)$ is $\<Matching>$-closed and therefore, 
$h'\setminus O_1\in H(\ker L)$. By the definition of $\preceq$, it follows that
$h_{c}\setminus O_1\preceq h'\setminus O_1$ and since the operations in $O_1$
are completed, we get that $h\setminus O_1\preceq h_{c}\setminus O_1$. Consequently,
$h\setminus O_1\in H(L)$.
\end{proof}

\begin{example}

Incompleteness because of removing complete and not obsolete operations.

\end{example}

\begin{example}

Incompleteness because we don't keep constraints deduced from the presence of the matching cluster.

\end{example}















