%!TEX root = head.tex

\section{Observational refinement}

\fbox{From POPL'15}

\subsection{Libraries}

Programs interact with libraries by calling named library \emph{methods}, which
receive \emph{parameter values} and yield \emph{return values} upon completion.
We fix arbitrary sets $\<Methods>$ and $\<Vals>$ of method names and
parameter/return values. 

\begin{example}
  \label{ex:methods}

  The method and value sets for the stack implementation in
  Figure~\ref{fig:treiber} are $\<Methods> = \set{ \<push>, \<pop> }$ and
  $\<Vals> = \<Nats> \u \set{ {\tt EMPTY} }$.

\end{example}

\noindent
We fix an arbitrary set $\<Ops>$ of operation identifiers, and for given sets
$\<Methods>$ and $\<Vals>$ of methods and values, we fix the sets
\begin{align*}
  & C = \set{ m(v)_o : m \in \<Methods>, v \in \<Vals>, o \in \<Ops> }
  \text{, and } \\
  & R = \set{ \<ret>(v)_o : v \in \<Vals>, o \in \<Ops> }  
\end{align*}
of \emph{call actions} and \emph{return actions}; each call action $m(v)_o$
combines a method $m \in \<Methods>$ and value $v \in \<Vals>$ with an
\emph{operation identifier} $o \in \<Ops>$. Operation identifiers are used to
pair call and return actions. We denote the operation identifier of a
call/return action $a$ by $\<op>(a)$. Call and return actions $c \in C$ and $r
\in R$ are \emph{matching}, written $c \match r$, when $\<op>(c) = \<op>(r)$. A
word $e \in @S^*$ over alphabet $@S$, such that $(C \u R) \subseteq @S$, is
\emph{well formed} when:
\begin{itemize}

  \item Each return is preceded by a matching call: \\
  $e_j \in R$ implies $e_i \match e_j$ for some $i < j$.

  \item Each operation identifier is used in at most one call/return: \\
  $\<op>(e_i) = \<op>(e_j)$ and $i < j$ implies $e_i \match e_j$.

\end{itemize}
We say that the well-formed word $e \in @S^*$ is \emph{sequential} when
\begin{itemize}

  \item Operations do not overlap: \\
  $e_i, e_k \in C$ and $i < k$ implies $e_i \match e_j$ for some $i < j < k$.

\end{itemize}
Well-formed words represent library executions. We assume every set of well-formed
words is closed under isomorphic renaming of operation identifiers. For
notational convenience, we often associate $\<Ops>$ with $\<Nats>$,
e.g.,~writing $m(u)_1$ and $\<ret>(v)_2$ in place of $m(u)_{o_1}$ and
$\<ret>(v)_{o_2}$. An operation $o$ of an execution $e$ is \emph{completed}
when both call and return actions $m(u)_o$ and $\<ret>(v)_o$ of $o$ occur in
$e$, and is otherwise \emph{pending}.
An execution with only completed operations is called \emph{complete}.

\begin{example}
  \label{ex:executions}

  The well-formed words
  \scriptsize
  \begin{align*}
     \<push>(0)_1\ \<pop>_2\ \<pop>_3\ \<ret>_1\ \<ret>(0)_3\ \<ret>(0)_2
    \text{\normalsize, and } 
    \<push>(0)_1\ \<pop>_2\ \<pop>_3\ \<ret>_1\ \<ret>(0)_2
  \end{align*}
  \normalsize
  represent executions in which one call to the $\<push>(0)$ method overlaps
  with two calls to $\<pop>$. In the first execution both calls to $\<pop>$
  have matching return actions $\<ret>(0)$, i.e., the operations $2$ and $3$ are completed,
  while operation $3$ is pending in the second, it has no matching return.

\end{example}

Libraries dictate the execution of methods between their call and return
points. Accordingly, a library cannot prevent a method from being called,
though it can decide not to return. Furthermore, any library action performed
in the interval between call and return points can also be performed should the
call have been made earlier, and/or the return made later. Our technical
results rely on these properties. A library thus allows any sequence of
invocations to its methods made by \emph{some} program.

\begin{definition}
  \label{def:library}

  A \emph{library} $L$ is a set of well-formed words over alphabet $C \u R$, called \emph{executions}, such that 
  \begin{itemize}

    \item Call actions $c \in C$ cannot be disabled: \\
    $e \cdot e' \in L$ implies $e \cdot c \cdot e' \in L$
    if $e \cdot c \cdot e'$ is well formed.
  
    \item Call actions $c \in C$ cannot disable other actions: \\
    $e \cdot a \cdot c \cdot e' \in L$ implies $e \cdot c \cdot a \cdot e' \in L$.
  
    \item Return actions $r \in R$ cannot enable other actions: \\
    $e \cdot r \cdot a \cdot e' \in L$ implies $e \cdot a \cdot r \cdot e' \in L$.
  
  \end{itemize}

\end{definition}

\noindent
We write $e_1 ~> e_2$ when $e_2$ can be derived from $e_1$ by applying zero or
more of the above rules. The \emph{closure} of a set $E$ of executions under
$~>$ is denoted $\overline{E}$. 

\begin{definition}
A minimal set of executions $\basis{L}\subseteq L$ such that 
$L = \overline{\basis{L}}$ is called a \emph{basis} of $L$. 
%and it is denoted by $\basis{L}$. 
\end{definition}

%If there exists a unique set of executions $E$ such that 
%$L = \overline{E}$, then $E$ is called \emph{the kernel of $L$}
%and it is denoted by $\ker L$. 



%The set of complete executions in $L$ is denoted by $L_c$.

Note that even a library that implements \emph{atomic methods}, e.g.,~by
guarding method bodies with a global-lock acquisition, admits executions in
which method calls and returns overlap. A library which accesses the client's
thread identifiers can be modeled by taking thread identifiers as method
parameters.

\begin{example}
  \label{ex:libraries}

  Any library which admits the execution
  \scriptsize
  \begin{align*}
    \<push>(0)_1\ \<ret>_1\ \<pop>_2\ \<ret>(0)_2
  \end{align*}
  \normalsize
  with sequential calls to $\<push>$ and $\<pop>$ must also admit
  \scriptsize
  \begin{align*}
    \<push>(0)_1\ \<pop>_2\ \<ret>_1\ \<ret>(0)_2
    \text{ \normalsize and }
    \<push>(0)_1\ \<pop>_2\ \<pop>_3\ \<ret>_1\ \<ret>(0)_2
    \text{\normalsize,}
  \end{align*}
  \normalsize
  among others, yet need not admit an execution
  \scriptsize
  \begin{align*}
    \<push>(0)_1\ \<pop>_2\ \<pop>_3\ \<ret>_1\ \<ret>(0)_3\ \<ret>(0)_2
  \end{align*}
  \normalsize
  with two completed $\<pop>$ operations returning $0$.
  
\end{example}

A library $L$ is called \emph{atomic} if there exists a basis $B$ of $L$ that consists only of
sequential executions (a basis that consists only of sequential executions is also unique). 
Note that $B$ contains only completed operations since $e_1
\cdot e_2 ~> e_1 \cdot e_2 \cdot c$, for any unmatched call $c$,
and $e_1 \cdot e_2 \cdot c\in L$ implies $e_1 \cdot e_2$ by the prefix closure of $L$.

Atomic libraries are often considered as specifications 
for concurrent objects.

\begin{example}
  \label{ex:atomic_stack}

  The \emph{atomic stack} is the library whose kernel is the set of sequential
  executions for which the return value of each $\<pop>$ operation is either
  \begin{itemize}
    
    \item the argument value $v$ to the last unmatched $\<push>$ operation, or
    
    \item {\tt EMPTY} if there are no unmatched $\<push>$ operations.

  \end{itemize}
  In practice, the atomic stack can be implemented by guarding the methods
  of a ``sequential'' stack object by global-lock acquisition.

\end{example}

\subsection{Histories}

For given sets $\<Methods>$ and $\<Vals>$ of methods and values, we fix a set
$\<Labels> = \<Methods> \x \<Vals> \x (\<Vals> \u \set{\bot})$ of
\emph{operation labels}, and denote the label $\tup{m,u,v}$ by $m(u) => v$. A
\emph{history} $h = \tup{O,<,f}$ is a partial order $<$ on a set $O \subseteq
\<Ops>$ of operation identifiers labeled by $f: O -> \<Labels>$ for which $f(o)
= m(u) => \bot$ implies $o$ is maximal in $<$. The \emph{history} $H(e)$ of
a well-formed execution $e \in @S^*$ labels each operation with a method-call
summary, and orders non-overlapping operations:
\begin{itemize}

  \item $O = \set{ \<op>(e_i) : 0 \le i < |e| \text{ and } e_i \in C }$,

  \item $\<op>(e_i) < \<op>(e_j)$ if{f} $i < j$, $e_i \in R$, and $e_j \in C$.

  \item $f(o) = \left\{
  \begin{array}{ll}
    m(u) => v     & \text{if } m(u)_o \in e \text{ and } \<ret>(v)_o \in e \\
    m(u) => \bot  & \text{if } m(u)_o \in e \text{ and } \<ret>(\_)_o \not\in e
  \end{array}
  \right.$

\end{itemize}
An operation of $h$ labeled by $@l\in \<Labels>$ is called an \emph{$@l$
operation}. The histories admitted by $L$ are $H(L) = \set{ H(e) : e \in L}$. 

The history of a complete execution is called \emph{complete}. 
The set of complete histories admitted by $L$ are $H_c(L)=\set{ H(e): e\in L,\mbox{ $e$ is complete}}$.

\begin{example}

  Figure~\ref{fig:stacks}(b) depicts the history of the execution in
  Figure~\ref{fig:stacks}(a). Arrows depict the order relation modulo
  transitivity. Operations $o_1$ and $o_2$ are ordered in $h$ if $o_1$'s return
  precedes $o_2$'s call. For example, $\<push>(1)$ precedes $\<pop>=>3$.
  However, $\<pop>=>1$ is incomparable to $\<pop>=>3$ because $\<pop>=>1$'s
  return comes after $\<pop>=>3$'s call, and $\<pop>=>3$'s return comes after
  $\<pop>=>1$'s call. The order among operations' call actions is irrelevant,
  as is the order among their return actions.

\end{example}

This notion of histories gives rise to a natural order relating histories by
their operation ordering. Basically, a history $h_1$ is \emph{weaker} than
another history $h_2$ if $h_2$ contains all completed operations of $h_1$, and
preserves the order between $h_1$'s operations. The pending operations of $h_1$
can be either omitted or completed in $h_2$.

\begin{definition}\label{def:weaker_than}

  Let $h_1 = \tup{O_1,<_1,f_1}$ and $h_2 = \tup{O_2,<_2,f_2}$. We say $h_1$ is
  \emph{weaker than} $h_2$, written $h_1 \preceq h_2$, when there exists an
  injection $g: O_2 -> O_1$ such that
  \begin{itemize}

    \item $o \in \<range>(g)$ when $f_1(o) = m(u) => v$ and $v \neq \bot$,

    \item $g(o_1) <_1 g(o_2)$ implies $o_1 <_2 o_2$ for each $o_1, o_2 \in O_2$,

    \item $f_1(g(o)) \ll f_2(o)$ for each $o \in O_2$.

  \end{itemize}
  where $(m_1(u_1) => v_1) \ll (m_2(u_2) => v_2)$ if{f} $m_1 = m_2$, $u_1 =
  u_2$, and $v_1 \in \set{ v_2, \bot }$. We say $h_1$ and $h_2$ are
  \emph{equivalent} when $h_1 \preceq h_2$ and $h_2 \preceq h_1$.

\end{definition}

\begin{example}

Figure~\ref{fig:stacks}(c) pictures a history $h'$ weaker than the history $h$ in Figure~\ref{fig:stacks}(b).
Note that $h'$ contains two pending $\<pop>=>\bot$ operations, one of them is completed in $h$ 
(it corresponds to the $\<pop>=>3$ operation) and one of them is omitted in $h$.

\end{example}

Throughout this work we do not distinguish between equivalent histories, and we
assume every set $H$ of histories is closed under inclusion of equivalent
histories, i.e.,~if $h_1$ and $h_2$ are equivalent and $h_1 \in H$, then $h_2
\in H$ as well.

\subsection{Refinement between Libraries}

Refinement between libraries is defined with respect to the observable actions
of programs which invoke library methods. Essentially, an implementation $L_1$ of a
concurrent object \emph{refines} another implementation $L_2$ if every
observable behavior of a program using $L_1$ is also observable using $L_2$.

\begin{definition}

  The library $L_1$ \emph{refines} $L_2$, written $L_1 \leq L_2$, iff $H(L_1)\subseteq H(L_2)$.
  
\end{definition}

