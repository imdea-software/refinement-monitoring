%!TEX root = head.tex
\section{Related Work}
\label{sec:related}

Previous work on automated refinement verification focuses on
\emph{linearizability}~\citep{journals/toplas/HerlihyW90}, which is now known
to be equivalent when considering atomic reference
implementations~\cite{journals/tcs/FilipovicORY10,conf/popl/BouajjaniEEH15}.
The theoretical limits of linearizability checking are well studied. While
checking a single execution is NP-complete~\cite{journals/siamcomp/GibbonsK97},
checking all executions of a finite-state implementation is in EXPSPACE when
the number of program threads is bounded~\cite{journals/iandc/AlurMP00}, and
undecidable otherwise~\cite{conf/esop/BouajjaniEEH13}.

Several semi-automated verification approaches rely on annotating method bodies
with \emph{linearization points}~\cite{conf/tacas/AbdullaHHJR13,
conf/cav/AmitRRSY07, conf/cav/DragoiGH13, conf/fm/LiuCLS09,
conf/podc/OHearnRVYY10, conf/cav/Vafeiadis10, conf/icse/Zhang11a} to reduce the
otherwise-exponential number of possible linearizations to one single
linearization. These methods typically rely on programmer annotation, and do
not admit conclusive evidence of a violation in the case of a failed proof.
``Aspect-oriented proofs'' reduce verification for certain atomic objects to a
small set of simpler properties. Specifically, checking executions of atomic
queue implementations against the theory $\textsc{Theory}(H_q)$ of the atomic
queue\footnote{Technically, the axioms of this theory without our totality
axiom.} \emph{kernel} from Section~\ref{sec:logic} is sound and complete for
complete histories~\cite{conf/concur/HenzingerSV13}.

% TODO either be clear on how this compares, or just omit it
% Another approach that partially avoids the need of user provided linearization
% points is described in~\cite{conf/issta/ShachamYGABSV14} where verifying
% linearizability of composed concurrent operations is reduced to finite state
% model checking.

Most automated approaches for detecting linearizability
violations~\cite{conf/pldi/BurckhardtDMT10, conf/asplos/BurnimNS11,
conf/kbse/ZhangCW13, journals/jpdc/WingG93} enumerate the
exponentially-many linearizations of each execution, limiting them to
executions with few operations, as observed in Section~\ref{sec:exp}.
Colt~\cite{conf/oopsla/ShachamBASVY11}'s approach mitigates this cost with
programmer-annotated linearization points, as mentioned above, and ultimately
suffers from the same problem: a failed proof only indicates incorrect
annotation.

One closely-related work aims to reduce the complexity of refinement checking
by approximating executions with weaker, bounded
histories~\cite{conf/popl/BouajjaniEEH15}. While this approach is also sound,
in the sense that only actual violations are flagged, its completeness in
practice is reported to rely on observing \emph{many} executions with varying
thread schedules, and its applicability is limited to executions with a bounded
number of argument/return values. On the contrary, the requirements for
long-term execution monitoring which we address demand completeness for each
individual execution without bounding the number of data values. Technically,
our approximations differ as well: while theirs forgets the order between some
operations, our match removal optimization forgets operations completely. Note
however that by removing operations only \emph{after} saturation of their
logical implications, the effect of forgotten operations can persist.
