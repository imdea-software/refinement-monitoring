%!TEX root = head.tex

%\section{Monitoring History Inclusion}

\section{Refinement via Propositional Reasoning}
\label{sec:propositional}

%A history formula describing the set of histories $H(B)$ 
%of a library basis $B$ is considered a specification for the entire library.
%
%Since every execution of a library $L$ can be obtained through the $~>$ rewriting 
%from an execution of a given basis $B$, 

%\begin{definition}
%
%A formula $\psi$ \emph{represents} $L$ iff there exists a basis $B$ of $L$ such that 
%for every history $h$, $h\in H(B)$ iff $h\models \psi$.
%
%\end{definition}

%Since $\psi$ is required to characterize only histories of a basis of $L$, there are
%histories of $L$ that don't satisfy $\psi$. However, for every history of $L$ there
%exists a stronger history which satisfies $\psi$.
Given a formula $\psi$ describing a basis of $L$, 
checking whether a history $h$ belongs to $H(L)$ 
can be reduced to propositional satisfiability. 
Thus, by Lemma~\ref{lemma:kernel_histories}, a history $h$ belongs to $H(L)$ iff 
it is weaker than a history $h'$ in a basis of $L$, or equivalently, weaker than a history
$h'$ satisfying $\psi$.
When $h$ is a complete history, the fact that any stronger history contains exactly the same set of 
operations~\footnote{More precisely, there exists a bijection between the set of operations in $h$ and respectively $h'$.}
enables the construction of a formula  $\mathit{Stronger}_c(h)$ characterizing all the histories
stronger than $h$. This formula conjuncted with $\psi$ describes all the histories stronger than $h$
that satisfy $\psi$. Therefore, the satisfiability of $\mathit{Stronger}_c(h)\land \psi$ is
equivalent to $h\in H(L)$. 
When the history $h$ contains pending operations, a stronger history $h'$ may contain
less operations (pending operations of $h$ can be omitted in $h'$)
and some pending operations of $h$ may be completed in $h'$.  
Therefore, the formula $\mathit{Stronger}_c(h)$ is enhanced with a domain predicate $\<ops>$
defining the set of operations in the stronger history and the return values of the pending
operations are left unconstrained. The conjunction between the obtained
formula and the formula $\psi'$ obtained from $\psi$ by guarding the quantified variables to satisfy
the domain predicate $\<ops>$ characterizes the histories stronger
than $h$ which satisfy $\psi$.

%$h\in H(L)$ is equivalent to the fact that $\mathit{Hist}(h)$ is consistent
%with the formula $\psi$ where the quantifiers are guarded by a domain predicate representing the
%set of operations in a history stronger than $h$.
%using the same
%function and predicate symbols as $\psi$. 
%using different versions $\<hb>_h$, $\<match>_h$, $\ldots$ of the predicates $\<hb>$, $\<match>$, $\ldots$, 
%and $\mathit{Weaker}$ is a formula expressing the fact that $\<hb>_h$ is included in $\<hb>$, 
%$\<match>_h$ is exactly $\<match>$, etc.

\subsection{Complete Histories}

The following two lemmas characterize the weaker than relation between histories. 
The first one states that the matching function of a history, projected over completed operations, 
is preserved by any stronger history. The second lemma states that a history $h'$ stronger than
a complete history $h$ can only differ in having more order constraints between the operations,
the operation labels and the matching function being the same in $h$ and $h'$.

\begin{lemma}

Let $h_1=\tup{O_1,<_1,f_1}$ and $h_2=\tup{O_2,<_2,f_2}$ be two histories. If $h_1\preceq_g h_2$ then
for every two completed operations $o_1, o_2\in O_1$, 
\[
\<Matching>(h_1)(o_1)=o_2\text{ iff }\<Matching>(h_2)(g^{-1}(o_1))=g^{-1}(o_2).
\]

\end{lemma}

\begin{lemma}\label{lemma:complete_history}

A complete history $h_1=\tup{O_1,<_1,f_1}$ is weaker than another history $h_2=\tup{O_2,<_2,f_2}$ 
iff there exists a bijection $g:O_2->O_1$ such that: %for every two operations $o, o'\in O_1$, 
\begin{itemize}

	\item operations related by $g$ have the same label, i.e., for each $o\in O_1$, $f_1(o)=f_2(g^{-1}(o))$,

	\item order constraints are preserved from $h_1$ to $h_2$, i.e., for each $o,o'\in O_1$, $o<_1 o'$ implies $g^{-1}(o)<_2 g^{-1}(o')$, and

	\item the matching function is the same for the two histories, i.e., for each $o,o'\in O_1$, $\<Matching>(h_1)(o)=o'$ iff $\<Matching>(h_2)(g^{-1}(o))=g^{-1}(o')$.

\end{itemize}

\end{lemma}

Given a complete history $h=\tup{O,<,f}$, let $\mathit{Stronger}_c(h)$ be a formula that characterizes
the labels, the matching function, and the order of $h$ (the operation identifiers in $O$ are used as constants):
\begin{align*}
\mathit{Stronger}_c(h)\ \triangleq\ & \forall x.\ x\in O\ \ \land \<Distinct>(O) \\
		    & \hspace{-13mm}\land\ \hspace{-2.5mm}\bigwedge_{o,o'\in O, o<o'} \hspace{-2.3mm}\<b>(o,o')\land \bigwedge_{\<Matching>(h)(o,o')} \<match>(o,o') \\
		    & \hspace{-13mm}\land\ \hspace{-7mm}\bigwedge_{\scriptsize \begin{array}{c} f(o)=m(u)=>v\end{array}} \hspace{-8mm} \<meth>(o)=m\land \<arg>(o)=u\land \<ret>(o)=v 
%		   \hspace{-7mm}\bigwedge_{\scriptsize \begin{array}{c} o,o'\in O\\f(o)=m(u)=>v\\f(o)=m'(v)=>v'\end{array}} \hspace{-7mm} \<Val>(o,o') \\
		   %\land \<PO>(<)
\end{align*}
The first sub-formula restricts the interpretation domain of every 
operation-identifier variable to $O$. The formula $x\in O$ is a syntactic sugar for 
$
\bigvee_{o\in O} x=o
$. 
The sub-formula $\<Distinct>(O)$ states that every two operation identifier constants are distinct,
i.e.,
$
\<Distinct>(O)\triangleq \bigwedge_{o_1,o_2\in O} o_1\neq o_2
$. 

%It uses indexed versions of the predicate and function symbols introduced in Section~\ref{sec:logic}
%The last sub-formula uses a relation symbol $\<hb>(o,o')$

%Also, let $\mathit{Weaker}$ be a formula expressing the weaker than relation between $h$ and another history 
%described using the standard symbols $\<hb>$, $\<meth>$, $\ldots$
%\begin{align*}
%\mathit{Weaker}\ \triangleq\ &\bigwedge_{F\in\{\<meth>,\<arg>,\<ret>\}} \forall x.\ F_h(x)=F(x) \\
%& \land\ \forall x,x'.\ \<hb>_h(x,x') => \<hb>(x,x')\\
%& \land\  \forall x,x'.\ \<match>_h(x,x') <=> \<match>(x,x')
%\end{align*}

%Also, let $\<Dom>(O)::=\forall x. x\in O$ be a


%Also, given $h$, let $F[h]$ be the HL formula $F$ where the relation symbol $<_k$ is replaced by $<$
%and the quantifiers are instantiated on the set of operations $O$, 
%i.e., every sub-formula $\exists x.\ G$ of $F$ is rewritten into
%\[
%\bigvee_{o\in O} G[x\mapsto o].
%\]

The following result states that $\mathit{Stronger}_c(h)$ characterizes precisely
all the histories stronger than $h$. It is a direct consequence of Lemma~\ref{lemma:complete_history}.

\begin{lemma}\label{lem:stronger_form_complete}

Let $h$ be a complete history. Then, for every history $h'$,
\begin{align*}
h\preceq h'\text{ iff }h'\models \mathit{Stronger}_c(h).
\end{align*}

\end{lemma}

Lemmas~\ref{lemma:kernel_histories} and~\ref{lem:stronger_form_complete} enable the following 
reduction from history membership to first-order satisfiability.

\begin{theorem}\label{th:satisfiability_pending}
Let $h$ be a complete history and $\psi$ a formula describing a basis of $L$.
Then,
\[
h\in H(L)\mbox{ iff }\mathit{Stronger}_c(h)\land \psi\mbox{ is satisfiable.}
\]
\end{theorem}
%\begin{proof}
%($=>$) This direction is a direct consequence of Lemmas~\ref{lemma:kernel_histories} and~\ref{lemma:complete_history}.
%
%%Let $h=\tup{O,<,f}$. Since $h\models \psi$, there exists a relation $\poker\subseteq O\times O$
%%such that $\tup{h,\poker}\models\psi'$, which implies that $\tup{h,\poker}$ is a model of $\varphi(h)\land \psi'$.
%
%($\Leftarrow$) The interpretation of $\<match>_h$, $\<hb>_h$, and $F_h$ with $F\in \{\<meth>,\<arg>,\<ret>\}$ 
%from a model of $\mathit{Hist}(h)\land\mathit{Weaker}\land \psi$ defines a history $h'$ such that $h\preceq h'$.
%On the other hand, the interpretation of $\<match>$, $\<hb>$, $\<meth>$, $\<arg>$, $\<ret>$ defines a history
%$h''$ such that $h'\preceq h''$ and $h''\models \psi$. Therefore, $h\preceq h''$ and $h''\models \psi$, which 
%by Lemma~\ref{lemma:kernel_histories} imply that $h\in H(L)$.
%\end{proof}

%\textcolor{red}{Say that there are no function symbols on input and output values, only = and $\leq$}

Assuming that the formula $\psi$ doesn't contain function symbols on input and output values but only
a fixed set of predicates, e.g., $=$ and $\leq$, which is the case for all the formulas in Figure~\ref{fig:examples_formulas}, 
the satisfiability of $\mathit{Stronger}_c(h)\land \psi$ 
can be reduced to propositional satisfiability.
This is based on the fact that $\mathit{Stronger}_c(h)$ fixes the domain of the operation-identifier variables to $O$
and every quantifier can be replaced by a finite conjunction/disjunction, e.g., every sub-formula $\exists x.\ A$ is replaced by 
$
\bigvee_{o\in O}\ A[x\mapsto o]
$.
Before eliminating the quantifiers, one also needs to add some universally-quantified axioms to 
express the fact that $\<b>$ is interpreted as a partial order and that $\<match>$ is interpreted as a 
partial function.
%existential quantifiers can be replaced by finite 
%disjunctions, i.e., every sub-formula $\exists x.\ A$ is replaced by 
%$
%\bigvee_{o\in O}\ A[x\mapsto o]
%$.
%The formula obtained by removing the existential quantifiers belongs to the EPR fragment
%(also known as the Bernays-Sch\"{o}nfinkel class),  whose satisfiability is known to be reducible to
%propositional satisfiability in polynomial time, provided that the quantifier count is fixed.
%The EPR fragment consists of first-order
%formulas with no occurrences of function symbols other than constants, and which  
%when written in the prenex normal form have the quantifier prefix $\exists^*\forall^*$.
%The formula $\varphi(h)\land \<Dom>(O)\land \psi'$ has no function symbols other than constants
%and by removing the existential quantifiers, it has a quantifier prefix $\forall^*$.
Therefore, one can construct a propositional formula $\mathit{Bool}_c(h,\psi)$, which
is equi-satisfiable to $\mathit{Stronger}_c(h)\land \psi$. The construction of
$\mathit{Bool}_c(h,\psi)$ can be done in polynomial time provided that the number of
quantifiers in $\psi$ is fixed, which is the case for all the formulas in Figure~\ref{fig:examples_formulas}.


\begin{corollary}\label{cor:satisfiability_complete}

Let $h$ be a complete history and $\psi$ a formula describing a basis of $L$. Then,
\[
h\in H(L)\mbox{ iff the formula}\ \mathit{Bool}_c(h,\psi)\mbox{ is satisfiable.}
\]

\end{corollary}

\subsection{Handling pending operations}\label{ssec:pending}

%The pending operations of a history $h$ may be omitted in a history stronger than $h$.
In the case of an arbitrary history $h$, the set of histories stronger than $h$ 
can be characterized by a formula obtained from $\mathit{Stronger}_c(h)$
by adding a domain predicate $\<ops>$
that is constrained to contain all the completed operations of $h$
and by omitting the constraints on the return values of pending operations.
Moreover, every operation of $h$ whose return value is different from $\bot$
in a model of this formula (this may be an operation which is pending in $h$) 
should satisfy $\<ops>$.
It can be proved that every history stronger than $h$ corresponds to a 
model of this formula, projected on the set of operations satisfying $\<ops>$.

Formally, let $\mathit{Stronger}(h)$ be the following formula:
\begin{align*}
\mathit{Stronger}(h)\ \triangleq\ & \forall x.\ x\in O\ \ \land \<Distinct>(O) \\
		    & \hspace{-10mm}\land\ \hspace{-2.5mm}\bigwedge_{o,o'\in O, o<o'} \hspace{-1.3mm}\<b>(o,o')\land \bigwedge_{\<Matching>(h)(o,o')} \<match>(o,o') \\
		    & \hspace{-10mm}\land\ \hspace{-7mm}\bigwedge_{\scriptsize \begin{array}{c} f(o)=m(u)=>v\end{array}} \hspace{-7mm} \<meth>(o)=m\land \<arg>(o)=u \\
		    & \hspace{-10mm}\land \ \hspace{-7mm}\bigwedge_{\scriptsize \begin{array}{c} f(o)=m(u)=>v\\ v\neq\bot\end{array}} \<ret>(o)=v\land \<ops>(o) \\
		    & \hspace{-10mm}\land \ \forall x.\ \<ret>(x)\neq\bot => \<ops>(x) %\bigwedge_{\scriptsize \begin{array}{c} f(o)=m(u)=>\bot\end{array}} \<ret>(o)\neq\bot => \<ops>(o)
\end{align*}

The models of $\mathit{Stronger}(h)$ are histories paired with an interpretation $D$ for the domain
predicate $\<ops>$. Given such a model $\tup{h',D}$, let $h'_D$ be the history obtained from $h'$
by removing all operations $o\not\in D$.

\begin{lemma}\label{lem:stronger_form_pending}

For every history $h$,
\begin{align*}
\tup{h',D}\models \mathit{Stronger}(h)\mbox{ iff }h\preceq h'_D
\end{align*}

\end{lemma}

The predicate $\<ops>$ is also used to guard the domain of the quantifiers in 
the formula $\psi$ defining a basis of the library $L$. For simplicity,
we assume that $\psi$ is given in prenex normal form and that it has a quantifier prefix 
of the form $\forall^*\exists^*$ (all the formulas in Figure~\ref{fig:examples_formulas} can 
be written in such a form). Thus, given $\psi=\forall \vec{x}\ \exists \vec{x}'.\ F$, 
let $\psi(\<ops>)$ be the formula:
\begin{align*}
\psi(\<ops>)\triangleq\forall \vec{x}\ \exists \vec{x}'.\ \bigwedge_{x\in\vec{x}}\ \<ops>(x) => \big(\bigwedge_{x'\in\vec{x}'}\ \<ops>(x') \land F\big)
\end{align*}
As usual, universal quantifiers are guarded using implication and existential quantifiers
using conjunction.



Lemmas~\ref{lemma:kernel_histories} and~\ref{lem:stronger_form_pending}
together with the guarded quantification added to $\psi$ imply that 
history membership can be reduced to first-order satisfiability.

\begin{theorem}\label{th:satisfiability_complete}
Let $h$ be a history and $\psi=\forall \vec{x}\ \exists \vec{x}'.\ F$ a formula describing a basis of $L$.
Then,
\[
h\in H(L)\mbox{ iff }\mathit{Stronger}(h)\land \psi(\<ops>)\mbox{ is satisfiable.}
\]
\end{theorem}


%The formula $\<PO>_c$ restricts the interpretation of the universally-quantified variables to
%complete operations:
%\<PO>_c(<)


The satisfiability of the formula $\mathit{Stronger}(h)\land \psi(\<ops>)$ can be again 
reduced to propositional satisfiability since the domain of the quantifiers is fixed to the set
of operations in $h$. Let $\mathit{Bool}(h,\psi)$ be the boolean formula equi-satisfiable
to $\mathit{Stronger}(h)\land \psi(\<ops>)$.

\begin{corollary}\label{cor:satisfiability_pending}

Let $h$ be a history and $\psi$ a formula describing a basis of $L$. Then,
\[
h\in H(L)\mbox{ iff the formula}\ \mathit{Bool}(h,\psi)\mbox{ is satisfiable.}
\]
\end{corollary}

