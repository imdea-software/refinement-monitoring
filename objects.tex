%!TEX root = head.tex

\section{Concurrent collections ??}

We define the class of concurrent libraries considered in this paper. 

\subsection{Matching Relations}

\newcommand{\domain}{\mathbb{D}}

We distinguish two sets of operation labels $\<PosLab> \subseteq \<Labels>$ and $\<NegLab> \subseteq \<Labels>$, called \emph{positive}, resp., \emph{negative}, operations labels.

\begin{example}

Positive and negative labels for different objects.

\begin{itemize}

\item{Queue}
  \begin{itemize}
  \item{Positive Labels:} $\set{\<enqueue>(x)\ |\ x \in \domain}$
  \item{Negative Labels:} $\set{\<dequeue> => x\ |\ x \in \domain}$
  \end{itemize}

\item{Stack}
  \begin{itemize}
  \item{Positive Labels:} $\set{\<push>(x)\ |\ x \in \domain}$
  \item{Negative Labels:} $\set{\<pop> => x\ |\ x \in \domain}$
  \end{itemize}

\item{Register}
  \begin{itemize}
  \item{Positive Labels:} $\set{\<write>(x)\ |\ x \in \domain}$
  \item{Negative Labels:} $\set{\<read> => x\ |\ x \in \domain}$
  \end{itemize}

\item{PriorityQueue}
  \begin{itemize}
  \item{Positive Labels:} $\set{\<add>(x)\ |\ x \in \domain}$
  \item{Negative Labels:} $\set{\<removeMax> => x\ |\ x \in \domain}$
  \end{itemize}
  
\item{Set}
  \todo{Necessary explanation for the choice we did for the set}
  \begin{itemize}
  \item{Positive Label:} $\<add>$
  \item{Negative Labels:} $\<remove>$ and $\<contains> => \top$
  \end{itemize}
  
\item{Lock}
  \begin{itemize}
  \item{Positive Label:} $\<lock>$
  \item{Negative Label:} $\<unlock>$
  \end{itemize}
  
\item{Semaphore}
  \begin{itemize}
  \item{Positive Label:} $\<acquire>$
  \item{Negative Label:} $\<release>$
  \end{itemize}

\item{Exchanger}
  \begin{itemize}
  \item{Positive Label:} $\<exchange>$
  \item{Negative Label:} $\<exchange>$
  \end{itemize}
  
\end{itemize}

\end{example}

We assume that each concurrent library comes with a matching relation $\<Matching>$ between negative and positive labels such that for every history $h$ of that library, each negative operation label in $h$ is matched to a unique positive label in $h$. Sometimes it is also required that every two different negative operation labels are matched to different positive labels.

\begin{definition}

A relation $\<Matching>\subseteq \<NegLab>\times \<PosLab>$ is a \emph{matching relation} for a library $L$ iff for every history $h=\tup{O,<,f}$ in $H(L)$, the following holds:
\begin{itemize}
	\item for every $o\in O$ such that $f(o)\in \<NegLab>$, there exists an unique $o'\in O$ such that $\<Matching>(f(o),f(o'))$.
\end{itemize}

A matching relation $\<Matching>$ is called \emph{injective} iff for every history $h=\tup{O,<,f}$ in $H(L)$ and $o_1\neq o_2\in O$ with $\{f(o_1),f(o_2)\}\in \<NegLab>$, there exist two different operations $o_1'\neq o_2'\in O$ such that $\<Matching>(f(o_1),f(o_1'))$ and $\<Matching>(f(o_2),f(o_2'))$.

\end{definition}

An operation $o$ is called \emph{negative}, resp., \emph{positive}, in a history $h=\tup{O,<,f}$ iff $f(o)\in \<NegLab>$, resp., $f(o)\in \<PosLab>$.

\begin{example}

We here give some examples of relations $\<Matching>$ and how they can be 
implemented. For usual implementations of objects which can be seen as 
containers, such as the Stack, the Queue, the Set, the Register, it is 
straightforward to instrument an implementation in order the get the matching 
relation. 

Each positive method (\<push>,\<enqueue>,\<add>,\<write>) is going to 
use a unique tag. When a negative method succeeds -- for instance pops an 
element from the stack -- it is also going to return the unique tag associated
with the element, thus giving us the relation $\<Matching>$.

\end{example}

\subsection{Kernel Closure Properties}

In this work, we consider the problem of checking refinement with respect to libraries $L$ for which $\ker L$ is defined.
The histories of executions in $\ker L$ are $H(\ker L)=\set{H(e) : e\in \ker L}$.

A \emph{prefix} of a history $h=\tup{O,<,f}$ is a history $h'=\tup{O',<',f'}$, that contains a subset of the operations in $h$,
i.e., $O'\subseteq O$, $<'=<|_{O'}$, and $f'=f|_{O'}$, and that is downward closed w.r.t. $<$, i.e., $o\in O'$ and $o'<o$ implies $o'\in O'$.

\begin{example}

Prefix of a history.

\end{example}

\begin{lemma}\label{lemma:kernel_histories_prefix}

Let $L$ be a library for which $\ker L$ is defined. The set of histories $H(\ker L)$ is prefix-closed.

\end{lemma}

Given a history $h=\tup{O,<,f}$, a set of operations $O'\subseteq O$ is called a \emph{matching cluster}
iff $O'$ consists of a single negative operation $o$ and all positive operations $o'$ with $\<Matching>(o,o')$.
A set of histories $H$ is closed under matching cluster removal iff $H$ contains all the histories
obtained by removing a matching cluster from another history in $H$. 
%We say that 
%a library $L$ is \emph{$\<Matching>$-closed} iff the set of kernel histories is closed under 
%matching cluster removal. 
Formally, given a history $h=\tup{O,<,f}$ and $O'\subseteq O$, 
\[
h\setminus O'=\tup{O\setminus O',<',f'},
\] 
where $<'=<|_{O\setminus O'}$, and $f'=f|_{O\setminus O'}$.

\begin{definition}

A set of histories $H$ is \emph{closed under matching cluster removal} 
(for short, $\<Matching>$-closed)
iff for every matching cluster $O'$ of a history $h\in H$,
the history $h\setminus O'$ also belongs to $H$.

\end{definition}

\begin{example}

Show that all objects we know satisfy this property.

\end{example}
