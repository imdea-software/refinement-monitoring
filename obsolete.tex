\section{Removing ``Obsolete'' Operations}
\label{sec:obsolete}

%Since from a theoretical point of view, testing history membership is NP-complete~\cite{journals/siamcomp/GibbonsK97},

Checking history membership for large histories using the reduction to propositional SAT 
from Corollary~\ref{cor:satisfiability_pending} can still be quite inefficient. 
One way of dealing with this inefficiency is to check history membership for a fragment
of the given history provided that the smaller history being a violation to
history inclusion implies that the initial history is also a violation. 
However, deleting arbitrary operations from a history $h\in H(L)$
may result in a history $h'\not\in H(L)$. For example, removing the $\<write>(1)$ operation
from the history of the atomic register pictured in Figure~\ref{fig:removal_even_saturation}
results in a history which is not anymore admitted by this object. To identify the
removable operations that preserve history membership we rely on the matching function.

A \emph{match} of a history $h$ is an operation $o$ together with the maximal 
set of operations mapped by $\<Matching>(h)$ to $o$. Moreover, a match consists only of 
completed operations and at least one operation mapped to $o$. Formally, a match of a 
history $h=\tup{O,<,f}$ is a set of operations $m=o\cup \<Matching>^{-1}(h)(o)$ such 
that $o\in O$, $\<Matching>^{-1}(h)(o)\neq \emptyset$, and all the operations in $m$
are completed.

\begin{example}

Example of a match.

\end{example}

A set of histories $H$ is \emph{match-removal closed} iff for every history $h\in H$
and every match $m$ of $h$, $H$ contains the history obtained from $h$ by deleting the 
operations in $m$. The history obtained from another history $h$ by deleting a set
of operations $O$ is denoted by $h\setminus O$.

The bases of all the reference implementations described in Figure~\ref{fig:examples_formulas} are 
match-removal closed. For instance, consider a sequential history in the basis of an atomic queue
$\sigma_1\cdot \<add>(1)\cdot\sigma_2\cdot \<rem>=>1\cdot\sigma_3$ 
(the history is written as a sequence of operation labels) such that the matching function 
maps the $\<rem>=>1$ operation to the $\<add>(1)$ operation. 
These two operations form a match
because the matching function for queue histories is injective (there is only one
operation mapped to $\<add>(1)$).
The history $\sigma_1\cdot \sigma_2\cdot \sigma_3$ obtained by removing the match 
is also a valid sequential queue history: the order between the remaining adds and 
removes is not affected by this removal.

\fbox{Need a general explanation for this closure}

\begin{theorem}

The bases of the atomic stacks, queues, priority queues, sets, locks, semaphores, and of the
exchanger are match-removal closed.

\end{theorem}

The match-removal closure property extends from a basis to the entire library.
Therefore, if by deleting matches from a history we get a history which is not admitted
by a library $L$ then the initial history is also not admitted by $L$.

\begin{theorem}\label{th:match_closure}

If a basis of a library $L$ is match-removal closed, then $H(L)$ is also match-removal closed.

\end{theorem}

\begin{proof}

Let $H(B)$ be a basis of $L$, $h$ a history in $H(L)$, and $m$
a match of $h$.
By Lemma~\ref{lemma:kernel_histories}, %and~\ref{lemma:pending_histories}, 
there exists a history $h'\in H(B)$ such that $h\preceq h'$. 
Since all the operations in $m$ are completed, they are included in $h'$ with the
same labeling. Moreover, the matching function projected over completed
operations is preserved from $h$ to $h'$. 
By hypothesis, $H(B)$ is match-removal closed which implies 
that the history $h_1'$ obtained from $h'$ by deleting the operations in $m$ 
is also in $H(B)$. From the definition of $\preceq$ it follows that
the history $h_1$ obtained from $h$ by deleting the operations in $m$ 
is weaker than $h_1'$ which implies $h_1\in H(L)$.
\end{proof}

While the statement of Theorem~\ref{th:match_closure} implies that a history
$h\setminus m$, where $m$ is a match of $h$, belongs to $H(L)$ whenever $h\in H(L)$,
Example~\ref{ex:complete_removal} shows that the reverse is not true. 
%Consequently, 
%a monitor that eagerly removes all matching clusters from the history it tracks 
%may miss some violations.

\begin{figure}

\input figures/complete_removal

\caption{A history that is not admitted by the atomic stack.}
\label{fig:complete_removal}

\end{figure}

\begin{example}\label{ex:complete_removal}

Figure~\ref{fig:complete_removal} pictures a history which is not admitted by the atomic stack:
since the element $3$ was pushed after $2$, $\<pop>=>3$ should not have started after 
$\<pop>=>2$ has finished.
%and $h'$, the latter being an
%extension of the former. 
The history obtained by removing the 
match $\{{\tt push}(2),{\tt pop}=>2\}$ is however admitted by the atomic stack.

\end{example}

When continuously checking membership to a reference implementation 
on a history that keeps extending with new operations, 
one must ensure that forgetting matches doesn't lead to spurious violations
at later times. In the continuation of Theorem~\ref{th:match_closure},
one should also prove that a match of  
a history $h$ is a match of every extension of $h$. In general, it can happen
is that a match of $h$ is strictly included in a match of an extension
and therefore, not a match of the extension.


Otherwise, removing a match before extending a history is equivalent to
removing a set of operations which is not a match from the extended history
(therefore, Theorem~\ref{th:match_closure} doesn't apply).
This result holds when the matching function of all histories in the library is injective
because every match has exactly two operations. All 
the reference implementations in Section~\ref{sec:logic} fall into this case 
except for the atomic sets and registers. 

A history $h_2$ \emph{extends} a history $h_1$, written $h_1 |> h_2$, iff $h_i$ is the history of an execution $e_i$, $i\in\{1,2\}$,
and $e_1$ is a prefix of $e_2$.

\begin{lemma}\label{lem:match_extension}

Let $L$ be a library s.t. $\<Matching>(h)$ is injective for all $h\in H(L)$.
For every two histories $h_1, h_2\in H(L)$ s.t. $h_1 |> h_2$,
if $m$ is a match of $h_1$ then $m$ is also a match of $h_2$.

\end{lemma}

When the matching function is not injective, 
a match $m$ of a history $h_1$ could be strictly included in a 
match of an extension $h_2$ which implies that $m$ is not a match of $h_2$. 

\begin{figure}

\input figures/removal_even_saturation

\caption{Two histories $h_1$ and $h_2$ of the atomic register, where $h_2$ is an extension of $h_1$.}
\label{fig:removal_even_saturation}

\end{figure}

\begin{example}\label{ex:removal_even_saturation}

Figure~\ref{fig:removal_even_saturation} pictures two histories $h_1$ and $h_2$ of the atomic
register, the latter being an extension of the former. 
The match $\{\<write>(1),\<read>=>1\}$ of $h_1$
is strictly included in the match $\{\<write>(1),\<read>=>1,\<read>=>1\}$ of $h_2$. 
Removing the match of $h_1$ and extending it with the $\<read>=>1$ operation
results in a spurious violation.

\end{example}

When the matching function is not injective



Lemma~\ref{lem:match_extension} and Theorem~\ref{th:match_closure} imply
that match removal is sound even 

\begin{corollary}

Let $L$ be a library with a match-removal closed basis
s.t. $\<Matching>(h)$ is injective for all $h\in H(L)$.
Then, for every two histories $h_1, h_2\in H(L)$ s.t. $h_1 |> h_2$ and
$m$ a match of $h_1$, 
$h_2\setminus m$ is a history of $L$.

\end{corollary}




Removing operations not in the matching relation, e.g. $\<pop>=>\<empty>$.

\newpage



Theorem~\ref{th:removing_matching} provides soundness guarantees for   
a monitor that instead of checking whether the tracked history belongs to $H(L)$,
checks whether a smaller history, obtained from $h$ by removing all matching clusters,
belongs to $H(L)$. In order to define a monitor that is allowed to remove
the matching clusters from the tracked history, either the matching relation $\<Matching>$
is injective or one needs to impose supplementary
conditions on the removed clusters. 
\begin{definition}

A matching cluster $O$ of a history $h$ is called \emph{obsolete} iff 

\begin{itemize}

	\item $\<Matching>$ is injective or

	\item there exists another matching cluster $O'$ of $h$ such that the positive operation in $O'$
	is obsolete and ordered after the positive operation in $O$

\end{itemize}

\end{definition}

\begin{lemma}

Let $L$ be a library such that for every two matching clusters $O_1$ and $O_2$ of a kernel history 
$h=\tup{O,<,f}$, $o_1<o_2$ for each $o_1\in O_1$ and $o_2\in O_2$. 
Also, let $h\in H(L)$ and $h'\in H(L)$ an extension of $h$. Then, there exists no matching cluster $O$ 
of $h'$, which strictly includes an obsolete matching cluster of $h$.

\end{lemma}




One way to avoid the incompleteness exhibited by Example~\ref{ex:complete_removal}
is to remove a matching cluster only if it consists of obsolete operations.
Example~\ref{ex:removal_no_saturation} shows that even this strategy is incomplete.
Intuitively, this happens because the monitor doesn't remember any constraints
on the kernel order implied by the presence of the matching cluster.

\begin{figure}

\input figures/removal_no_saturation

\caption{Removing the matching cluster $\{{\tt push}(1),{\tt pop}=>1\}$ formed of obsolete operations.}
\label{fig:removal_no_saturation}

\end{figure}

\begin{example}\label{ex:removal_no_saturation}

Figure~\ref{fig:removal_no_saturation} pictures two histories $h$ and $h'$, the latter being an
extension of the former. Even though the matching cluster $\{{\tt push}(1),{\tt pop}=>1\}$
consists only of obsolete operations, removing it from the history $h'\not\in H(L_{stacks})$ 
results in a history $h$ which is not anymore a violation, since $h\in H(L_{stacks})$.
In order to detect this violation, the monitor should remember the constraint
$\<push>(2) \poker \<pop>=>Empty$ implied by the presence of this matching cluster.

\end{example}


