%!TEX root = conf-pldi-EmmiEH15.tex
\section{Removing Matched Operations}
\label{sec:obsolete}

While the reduction to symbolic reasoning enabled by the previous sections
already offers practical advantages over the explicit enumeration of history
linearizations, this reduction does nothing to avoid the increasing cost of
refinement checking as execution-length increases. A truly useful runtime
monitor must be \emph{linear} in the number of operations in order to avoid a
progressive slowdown of the monitored implementation. Achieving this complexity
goal implies forgetting increasingly-many previously-executed operations.
However, forgetting arbitrary operations from valid histories can result in
falsely-reported violations. For example, removing the write$(1)$
operation from the valid atomic register history of
Figure~\ref{fig:removal_even_saturation} results in an invalid history. We
leverage the same notion of operation matching used to characterize library
kernels in order to identify groups of operations whose removal preserves
history membership.

\subsection{Unique matching functions}

Recall that a history $h\in \overline{H}$ iff there exists a history $h'$
stronger than $h$ and a matching function $M : O \rightharpoonup O$ such that
$h',M\models \textsc{Theory}(H)$. We assume in the following that the 
matching function $M : O \rightharpoonup O$ of every history $h$ is uniquely determined 
by the operation labels in $h$, i.e.,~there exists a function
$\mathsf{M}:\mathbb{L} \rightharpoonup \mathbb{L}$ such that $M(o)=o'$ iff
$\mathsf{M}(f(o))=f(o')$. The matching function associated to some history $h$
is denoted by $M_h$. We give some examples of how to construct histories of
standard libraries with unique matching functions.

\begin{example}[Collections]

  For usual implementations of collections such as stacks, queues, and sets,
  each operation adding a value to the collection is going to receive as input
  a value which is uniquely identified by a tag. When a method removing an
  element from the collection succeeds it is also going to return the unique
  tag associated with that element, thus defining a unique matching function
  from remove operations to add operations. The same strategy can be adapted to
  implementations of a register: the inputs to write operations are tagged and
  the read operations return tagged values.

\end{example}

\begin{example}[Locks]

  The implementations of a lock object usually have two abstract states, one
  where the object is unlocked, and one where it is locked. The lock operations
  can be modified to receive as input a value which is unique for every lock
  operation in an execution and every successful execution of a lock operation
  results in an object state that stores that input value. When an unlock
  operation succeeds, it returns the value stored in the object state.
  Therefore, the matching function maps (\textnormal{unlock} $\Rightarrow v$) operations to
  \textnormal{lock}$(v)$ operations.

\end{example}

\begin{example}[Semaphores]

  Semaphore objects are usually implemented using a counter, which counts the
  number of acquire operations which successfully entered the semaphore and
  which are not yet released. We can instrument the implementation by keeping a
  map which maps each \emph{slot} (from $1$ to the capacity $c$) to a unique
  tag which was received as input by the acquire operation which has that slot,
  if any. When a release succeeds in decrementing the counter, it returns the
  unique tag of the acquire which had the slot.

\end{example}

\subsection{Closure under removing matched operations}

A \emph{match} of a history $h$ is an operation $o$ together with the maximal 
set of operations mapped by $M_h$ to $o$. 
Moreover, a match consists only of 
completed operations and at least one operation mapped to $o$. Formally, a match of a 
history $h=\tup{O,<,f}$ is a set of operations $m=o\cup M^{-1}_h(o)$ such 
that $o\in O$, $M^{-1}_h(o)\neq \emptyset$, and all the operations in $m$
are completed. The operation $o$ of a match $m=o\cup M^{-1}_h(o)$ 
is denoted by $+(m)$.

\begin{example}

  Let $h$ be the history in Figure~\ref{fig:complete_removal} such that the
  matching function $M_h$ maps every (\textnormal{rem} $\Rightarrow i$) operation to the
  \textnormal{add}$(i)$ operation. The matches of $h$ are $m_1 = \set{ \textnormal{add}(1),
  \textnormal{rem}\Rightarrow 1 }$, $m_2 = \set{ \textnormal{add}(2),
  \textnormal{rem}\Rightarrow 2 }$, and $m_3 = \set{ \textnormal{add}(3),
  \textnormal{rem}\Rightarrow 3 }$. Since every operation has a different label, we
  abuse the notation and write matches as sets of operation labels. Then,
  $+(m_i) = \textnormal{add}(i)$, for each $i$.

\end{example}

A set of histories $H$ is \emph{match-removal closed} iff for every history
$h\in H$ and every match $m$ of $h$, $H$ contains the history obtained from $h$
by deleting the operations in $m$. The history obtained from another history
$h$ by deleting a set of operations $O$ is denoted by $h \setminus O$.

The kernels of all the reference implementations described in
Section~\ref{sec:logic} are match-removal closed. For instance, consider a
sequential history $h$ in the basis of an atomic queue and a match $m=\set{
\text{add}(1), \text{rem}\Rightarrow 1 }$. The history obtained from $h$ by
removing the match $m$ is also a valid sequential queue history because
essentially, the remaining values are still removed in the order in which they
are added. Match-removal closure follows from the fact that for every such kernel 
$H$, the formula $\textsc{Theory}(H)$ holds for $h\setminus m$ whenever it holds for $h$,
for any history $h$ and $m$ an arbitrary match of $h$.

\begin{theorem}
The kernels of the atomic queue, stack, set, register, lock, semaphore, and exchanger are 
match-removal closed.
\end{theorem}

The match-removal closure property extends from a kernel $H$ to the entire
library $\overline{H}$. Therefore, if by deleting matches from a history we get
a history which is not admitted by a library $\overline{H}$ then the initial
history is also not admitted by $\overline{H}$.

\begin{theorem}
  \label{th:match_closure}

  $\overline{H}$ is match-removal closed if $H$ is.

\end{theorem}

\begin{proof}

  Let $h\in \overline{H}$ and $m$ a match of $h$. By definition, there exists a
  history $h'\in H$ such that $h\preceq h'$. 
  Since the matching functions are uniquely determined by the operation labels, 
  $m$ is also a match of $h'$. By hypothesis, $H$ is match-removal
  closed which implies that the history $h''$ obtained from $h'$ by deleting
  the operations in $m$ is also in $H$. From the definition of $\preceq$ it
  follows that the history $h\setminus m$ is weaker than $h''$ which implies
  $h\setminus m\in \overline{H}$.
  %
\end{proof}

While the statement of Theorem~\ref{th:match_closure} implies that a history
$h\setminus m$, where $m$ is a match of $h$, belongs to $\overline{H}$ whenever
$h\in \overline{H}$, Example~\ref{ex:complete_removal} shows that the reverse
is not true.

\begin{figure}

  \input figures/complete_removal

  \caption{A history that is not admitted by the atomic stack. Each operation
  is represented by an horizontal line segment. The line segment of an
  operation ending before the line segment of another operation means that the
  two operations are ordered (reading from left to right).}
  \label{fig:complete_removal}

\end{figure}

\begin{example}
  \label{ex:complete_removal}

  Figure~\ref{fig:complete_removal} pictures a history $h$ which is not admitted by
  the atomic stack: since the element $3$ was pushed after $2$, (\textnormal{rem} $\Rightarrow
  3$) should not have started after (\textnormal{rem} $\Rightarrow 2$) has finished. The
  matching function $M_h$ maps every (\textnormal{rem} $\Rightarrow i$) operation to the
  \textnormal{add}$(i)$ operation.

  The history obtained by removing the match $\set{ \textnormal{add}(2),
  \textnormal{rem}\Rightarrow 2 }$ is however admitted by the atomic stack.

\end{example}

\subsection{Forgetting matched operations}

A runtime monitor continuously checking history membership can soundly forget
matches provided that Theorem~\ref{th:match_closure} holds and that every match of a history $h$ is also a match of 
every extension of $h$ with new operations (a match of $h$
could be strictly included in a match of an extension and therefore, not a
match of the extension). If the latter doesn't hold, then removing an arbitrary match before
extending a history may be the same as removing a set of operations which is
not a match from the extended history (and
Theorem~\ref{th:match_closure} wouldn't apply). When the
matching function of all the histories in the library is injective, every
match has exactly two operations and this property trivially holds. All the 
reference implementations in
Section~\ref{sec:logic} have injective matching functions except for the atomic sets and
registers. Nevertheless, we prove in the following that the atomic sets and registers also 
satisfy this property (that every match of a history $h$ is also a match of 
every extension of $h$).

%When continuously checking history membership on a
%history that keeps extending with new operations, one must ensure that
%forgetting matches doesn't lead to spurious violations at later times. In the
%continuation of Theorem~\ref{th:match_closure}, one should also prove that a
%match of a history $h$ is a match of every extension of $h$ (a match of $h$
%could be strictly included in a match of an extension and therefore, not a
%match of the extension). Otherwise, removing an arbitrary match before
%extending a history may be the same as removing a set of operations which is
%not a match from the extended history (therefore,
%Theorem~\ref{th:match_closure} wouldn't apply). This result holds when the
%matching function of all histories in the library is injective because every
%match has exactly two operations. All the reference implementations in
%Section~\ref{sec:logic} fall into this case except for the atomic sets and
%registers.

A history $h_2$ \emph{extends} $h_1$ iff $h_2$ is the history
of an execution that extends the execution represented by $h_1$. Formally, $h_2
= \tup{O_2,<_2,f_2}$ extends $h_1 = \tup{O_1,<_1,f_1}$, written $h_1
\vartriangleright h_2$, iff
\begin{itemize}

	\item $O_1\subseteq O_2$, 
	
	\item $f_1(o) \ll f_2(o)$ for each $o \in O_1$,
	
	\item $o_1 <_1 o_2$ iff $o_1 <_2 o_2$ for each $o_1,o_2\in O_1$, and
	
	\item $o_1 <_2 o_2$ for each $o_1\in O_1$ and $o_2\in O_2\setminus O_1$.

\end{itemize}

\begin{lemma}
  \label{lem:match_extension1}

  Let $\overline{H}$ be a library such that $M_h$ is injective for all $h\in
  \overline{H}$. For every two histories $h_1, h_2\in \overline{H}$ such that
  $h_1 \vartriangleright h_2$, if $m$ is a match of $h_1$ then $m$ is also a
  match of $h_2$.

\end{lemma}

Example~\ref{ex:removal_even_saturation} shows that this result doesn't hold
when the matching function is not injective.

\begin{figure}

  \input figures/removal_even_saturation

  \caption{Two histories $h_1$ and $h_2$ of the atomic register, where $h_2$ is
  an extension of $h_1$.}
  \label{fig:removal_even_saturation}

\end{figure}

\begin{example}
  \label{ex:removal_even_saturation}

  Figure~\ref{fig:removal_even_saturation} pictures two histories $h_1$ and
  $h_2$ of the atomic register, the latter being an extension of the former. We
  assume that the matching function maps every (\textnormal{read} $\Rightarrow i$) operation to
  the \textnormal{write}$(i)$ operation. The match $\set{ \textnormal{write}(1),
  \textnormal{read}\Rightarrow 1 }$ of $h_1$ is strictly included in the match $\set{
  \textnormal{write}(1), \textnormal{read}\Rightarrow 1, \textnormal{read}\Rightarrow 1 }$ of
  $h_2$, and therefore not a match of $h_2$. Removing the match of $h_1$ and
  extending it with the (\textnormal{read} $\Rightarrow 1$) operation results in a spurious
  violation.

\end{example}

For libraries with non-injective matching functions, we identify conditions on their kernels 
which imply that a match $m$ of a history $h$ is a match of
every extension of $h$.

Let $R$ be a relation on operation labels. We say that a history $h$ is
\emph{$R$-ordered} iff for any two matches $m_1$ and $m_2$ of $h$ such that
$R(+(m_1), +(m_2))$, each operation of $m_1$ is ordered before each operation
of $m_2$.

Consider the atomic register and the relation $R_\mathrm{reg}$ which holds for
every two labels of two write operations. Then, any history from its kernel
(which by definition is sequential) is $R$-ordered because every read operation
is mapped by the matching function to the closest preceding write operation.
Similarly, one can show that any history from the kernel of the atomic set is
$R_\mathrm{set}$-ordered, where $R_\mathrm{set}$ holds for every two labels
add$(x)$ and add$(y)$ with $x = y$.

A set of histories $H$ is $R$-ordered iff every history in $H$ is $R$-ordered.
Let $\overline{H}$ be a library such that $H$ is $R$-ordered. A match $m$ of a
history $h\in \overline{H}$ is \emph{overwritten} by another match $m'$ of $h$
iff the labels of $+(m)$ and $+(m')$ are related by $R$, $+(m)$ finishes before
$+(m')$, and every operation overlapping with $+(m')$ is completed.

\begin{example}

  Given the histories $h_1$ and $h_2$ in
  Figure~\ref{fig:removal_even_saturation}, the match $\set{ \textnormal{write}(0),
  \textnormal{read}\Rightarrow 0, \textnormal{read}\Rightarrow0 }$ is overwritten by the
  match $\set{ \textnormal{write}(1), \textnormal{read}\Rightarrow 1,
  \textnormal{read}\Rightarrow1 }$ in $h_2$ but not in $h_1$ since a (\textnormal{read} $\Rightarrow
  0$) operation is pending in $h_1$.

\end{example}

Given a match $m$ overwritten by another match $m'$ in $h\in \overline{H}$,
every new completed operation $o$ from an extension $h'$ of $h$ cannot be
matched to $+(m)$, therefore $m$ is also a match of $h'$. Otherwise, since all
the operations overlapping with $+(m')$ are completed in $h$, $o$ starts after
$+(m')$ and $H$ would contain a history where the operations $+(m)$, $+(m')$,
$o$ occur in this order. Therefore, $H$ is not $R$-ordered.

\begin{lemma}
  \label{lem:match_extension2}

  Let $H$ be a $R$-ordered, and $h_1, h_2\in \overline{H}$. If $h_1 \vartriangleright h_2$ and
  $m$ is a match of $h_1$ overwritten by another match $m'$ of $h_1$, then $m$
  is also a match of $h_2$.

\end{lemma}

A match $m$ of a history $h\in \overline{H}$ is called \emph{obsolete} if
(1)~it is simply a match when the matching function of every history $h\in
\overline{H}$ is injective or (2)~it is overwritten by another match of $h$,
when $H$ is $R$-ordered. Lemmas~\ref{lem:match_extension1}
and~\ref{lem:match_extension2}, and Theorem~\ref{th:match_closure} imply the
following.

\begin{corollary}
  \label{cor:matching_final}

  Let $H$ be a match-removal closed history set such that $M_h$ is injective
  for all $h\in \overline{H}$ or $H$ is $R$-ordered, for some $R$. For every
  two histories $h_1, h_2\in \overline{H}$ such that $h_1 \vartriangleright
  h_2$ and $m$ an obsolete match of $h_1$, $h_2\setminus m$ is a history of
  $\overline{H}$.

\end{corollary}

The following illustrates a possible source of incompleteness for monitoring
algorithms which remove operations.

\begin{example}
  \label{ex:removal_no_saturation}

  Figure~\ref{fig:removal_no_saturation} pictures two histories $h_1$ and
  $h_2$, the latter being an extension of the former. The (\textnormal{rem} $\Rightarrow
  2$)-operation is pending in $h_1$ and only completes in $h_2$. However,
  removing the match $\set{ \textnormal{add}(1), \textnormal{rem}\Rightarrow 1 }$ from $h_1$
  before the pending \textnormal{rem} completes results in a history admitted by the
  atomic stack. This highlights a risk of incompleteness which we must account
  for in our development of refinement-monitoring algorithms. Should we simply
  forget about the match $\set{ \textnormal{add}(1), \textnormal{rem} \Rightarrow 1}$ while
  monitoring, we may not detect the violation obviated when the pending
  \textnormal{rem} completes. In order to avoid such incompleteness, at the very least,
  our monitoring algorithm must remember that the \textnormal{add}$(2)$ operation
  should be ordered before \textnormal{rem} $\Rightarrow {\tt empty}$ (since
  \textnormal{add}$(2)$ must be ordered before \textnormal{rem} $\Rightarrow 1$, which is ordered before
  \textnormal{rem} $\Rightarrow {\tt empty}$), even after removing the match. Then
  when (\textnormal{rem} $\Rightarrow 2$) completes, we could deduce that (\textnormal{rem}
  $\Rightarrow {\tt empty}$) must be ordered after \textnormal{add}$(2)$ and before (\textnormal{rem}
  $\Rightarrow 2$), a contradiction with the atomic stack theory
  $\textsc{Theory}(H_{\sf st})$ of Section~\ref{sec:logic}.

\end{example}

\begin{figure}
  \input figures/removal_no_saturation
  \caption{Two histories which are not admitted by the atomic stack, $h_2$
  being an extension of $h_1$.}
  \label{fig:removal_no_saturation}
\end{figure}
